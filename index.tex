% Options for packages loaded elsewhere
\PassOptionsToPackage{unicode}{hyperref}
\PassOptionsToPackage{hyphens}{url}
%
\documentclass[
  letterpaper,
]{scrbook}

\usepackage{amsmath,amssymb}
\usepackage{iftex}
\ifPDFTeX
  \usepackage[T1]{fontenc}
  \usepackage[utf8]{inputenc}
  \usepackage{textcomp} % provide euro and other symbols
\else % if luatex or xetex
  \usepackage{unicode-math}
  \defaultfontfeatures{Scale=MatchLowercase}
  \defaultfontfeatures[\rmfamily]{Ligatures=TeX,Scale=1}
\fi
\usepackage{lmodern}
\ifPDFTeX\else  
    % xetex/luatex font selection
\fi
% Use upquote if available, for straight quotes in verbatim environments
\IfFileExists{upquote.sty}{\usepackage{upquote}}{}
\IfFileExists{microtype.sty}{% use microtype if available
  \usepackage[]{microtype}
  \UseMicrotypeSet[protrusion]{basicmath} % disable protrusion for tt fonts
}{}
\makeatletter
\@ifundefined{KOMAClassName}{% if non-KOMA class
  \IfFileExists{parskip.sty}{%
    \usepackage{parskip}
  }{% else
    \setlength{\parindent}{0pt}
    \setlength{\parskip}{6pt plus 2pt minus 1pt}}
}{% if KOMA class
  \KOMAoptions{parskip=half}}
\makeatother
\usepackage{xcolor}
\setlength{\emergencystretch}{3em} % prevent overfull lines
\setcounter{secnumdepth}{5}
% Make \paragraph and \subparagraph free-standing
\ifx\paragraph\undefined\else
  \let\oldparagraph\paragraph
  \renewcommand{\paragraph}[1]{\oldparagraph{#1}\mbox{}}
\fi
\ifx\subparagraph\undefined\else
  \let\oldsubparagraph\subparagraph
  \renewcommand{\subparagraph}[1]{\oldsubparagraph{#1}\mbox{}}
\fi


\providecommand{\tightlist}{%
  \setlength{\itemsep}{0pt}\setlength{\parskip}{0pt}}\usepackage{longtable,booktabs,array}
\usepackage{calc} % for calculating minipage widths
% Correct order of tables after \paragraph or \subparagraph
\usepackage{etoolbox}
\makeatletter
\patchcmd\longtable{\par}{\if@noskipsec\mbox{}\fi\par}{}{}
\makeatother
% Allow footnotes in longtable head/foot
\IfFileExists{footnotehyper.sty}{\usepackage{footnotehyper}}{\usepackage{footnote}}
\makesavenoteenv{longtable}
\usepackage{graphicx}
\makeatletter
\def\maxwidth{\ifdim\Gin@nat@width>\linewidth\linewidth\else\Gin@nat@width\fi}
\def\maxheight{\ifdim\Gin@nat@height>\textheight\textheight\else\Gin@nat@height\fi}
\makeatother
% Scale images if necessary, so that they will not overflow the page
% margins by default, and it is still possible to overwrite the defaults
% using explicit options in \includegraphics[width, height, ...]{}
\setkeys{Gin}{width=\maxwidth,height=\maxheight,keepaspectratio}
% Set default figure placement to htbp
\makeatletter
\def\fps@figure{htbp}
\makeatother
% definitions for citeproc citations
\NewDocumentCommand\citeproctext{}{}
\NewDocumentCommand\citeproc{mm}{%
  \begingroup\def\citeproctext{#2}\cite{#1}\endgroup}
\makeatletter
 % allow citations to break across lines
 \let\@cite@ofmt\@firstofone
 % avoid brackets around text for \cite:
 \def\@biblabel#1{}
 \def\@cite#1#2{{#1\if@tempswa , #2\fi}}
\makeatother
\newlength{\cslhangindent}
\setlength{\cslhangindent}{1.5em}
\newlength{\csllabelwidth}
\setlength{\csllabelwidth}{3em}
\newenvironment{CSLReferences}[2] % #1 hanging-indent, #2 entry-spacing
 {\begin{list}{}{%
  \setlength{\itemindent}{0pt}
  \setlength{\leftmargin}{0pt}
  \setlength{\parsep}{0pt}
  % turn on hanging indent if param 1 is 1
  \ifodd #1
   \setlength{\leftmargin}{\cslhangindent}
   \setlength{\itemindent}{-1\cslhangindent}
  \fi
  % set entry spacing
  \setlength{\itemsep}{#2\baselineskip}}}
 {\end{list}}
\usepackage{calc}
\newcommand{\CSLBlock}[1]{\hfill\break\parbox[t]{\linewidth}{\strut\ignorespaces#1\strut}}
\newcommand{\CSLLeftMargin}[1]{\parbox[t]{\csllabelwidth}{\strut#1\strut}}
\newcommand{\CSLRightInline}[1]{\parbox[t]{\linewidth - \csllabelwidth}{\strut#1\strut}}
\newcommand{\CSLIndent}[1]{\hspace{\cslhangindent}#1}

\makeatletter
\@ifpackageloaded{bookmark}{}{\usepackage{bookmark}}
\makeatother
\makeatletter
\@ifpackageloaded{caption}{}{\usepackage{caption}}
\AtBeginDocument{%
\ifdefined\contentsname
  \renewcommand*\contentsname{Table of contents}
\else
  \newcommand\contentsname{Table of contents}
\fi
\ifdefined\listfigurename
  \renewcommand*\listfigurename{List of Figures}
\else
  \newcommand\listfigurename{List of Figures}
\fi
\ifdefined\listtablename
  \renewcommand*\listtablename{List of Tables}
\else
  \newcommand\listtablename{List of Tables}
\fi
\ifdefined\figurename
  \renewcommand*\figurename{Figure}
\else
  \newcommand\figurename{Figure}
\fi
\ifdefined\tablename
  \renewcommand*\tablename{Table}
\else
  \newcommand\tablename{Table}
\fi
}
\@ifpackageloaded{float}{}{\usepackage{float}}
\floatstyle{ruled}
\@ifundefined{c@chapter}{\newfloat{codelisting}{h}{lop}}{\newfloat{codelisting}{h}{lop}[chapter]}
\floatname{codelisting}{Listing}
\newcommand*\listoflistings{\listof{codelisting}{List of Listings}}
\makeatother
\makeatletter
\makeatother
\makeatletter
\@ifpackageloaded{caption}{}{\usepackage{caption}}
\@ifpackageloaded{subcaption}{}{\usepackage{subcaption}}
\makeatother
\ifLuaTeX
  \usepackage{selnolig}  % disable illegal ligatures
\fi
\usepackage{bookmark}

\IfFileExists{xurl.sty}{\usepackage{xurl}}{} % add URL line breaks if available
\urlstyle{same} % disable monospaced font for URLs
\hypersetup{
  pdftitle={Explaining Atheism: Wave 2 Codebook},
  hidelinks,
  pdfcreator={LaTeX via pandoc}}

\title{Explaining Atheism: Wave 2 Codebook}
\author{}
\date{}

\begin{document}
\frontmatter
\maketitle

\renewcommand*\contentsname{Table of contents}
{
\setcounter{tocdepth}{2}
\tableofcontents
}
\mainmatter
\bookmarksetup{startatroot}

\chapter*{Welcome}\label{welcome}
\addcontentsline{toc}{chapter}{Welcome}

\markboth{Welcome}{Welcome}

\texttt{**\ This\ is\ a\ living\ document\ as\ is\ not\ necessarily\ complete.**}

\texttt{A\ timestamped\ final\ codebook\ will\ be\ published\ alongside\ the\ dataset\ when\ this\ is\ released}

Welcome to the codebook for the second of the Explaining Atheism
project. Here you will find all the information about what is in the
accompanying dataset. This is a more focussed examination of variables
tested in wave 1 (see here) across more countries.

\subsection*{Data}\label{data}
\addcontentsline{toc}{subsection}{Data}

This dataset is being collected via international surveys in Brazil,
Japan, China, the United States, Denmark annd the UK.

The dataset is comprised of two surveys of possible explanations of
belief and non-belief, with the same belief measures included in both
surveys. Each survey is run in a nationally representative sample (age,
sex, region) of 1250 participants in each of the six countries.

\subsection*{Additional Information}\label{additional-information}
\addcontentsline{toc}{subsection}{Additional Information}

All information about the project, including scorings scripts for each
measures used, is available on our Figshare page upon release, along
with on my personal GitHub. This includes links to pre-registrations,
materials and the data itself.

If you have any questions or want any further information feel free to
contact me at c.russell@qub.ac.uk, or get in touch via my GitHub or
Twitter linked on the top left!

\part{Survey 1}

\chapter{Rebelliousness}\label{rebelliousness}

\textbf{Cluster:} Morals/Values

\section{Measure}\label{measure}

The rebelliousness measure is a custom measure devised to examine the
subscales pro-active rebelliousness and reactive rebelliousness
identified by McDermott (2001). After wave 1 we exluded what we termed
the ``Activism'' subscale and only included the ``Trolling'' subscale in
this round.

\subsection*{Modifications}\label{modifications}
\addcontentsline{toc}{subsection}{Modifications}

\section{Implementation}\label{implementation}

\subsection*{Question wording}\label{question-wording}
\addcontentsline{toc}{subsection}{Question wording}

Participants read the following text:

\emph{Please read each of the following statements carefully and state
to what extent they apply to you. There are no right or wrong answers
and your responses remain anonymous.}

\subsection*{Items}\label{items}
\addcontentsline{toc}{subsection}{Items}

\begin{longtable}[]{@{}
  >{\raggedright\arraybackslash}p{(\columnwidth - 4\tabcolsep) * \real{0.2329}}
  >{\raggedright\arraybackslash}p{(\columnwidth - 4\tabcolsep) * \real{0.2329}}
  >{\raggedright\arraybackslash}p{(\columnwidth - 4\tabcolsep) * \real{0.5342}}@{}}
\toprule\noalign{}
\begin{minipage}[b]{\linewidth}\raggedright
Qlabel
\end{minipage} & \begin{minipage}[b]{\linewidth}\raggedright
\textbf{Subscale}
\end{minipage} & \begin{minipage}[b]{\linewidth}\raggedright
question
\end{minipage} \\
\midrule\noalign{}
\endhead
\bottomrule\noalign{}
\endlastfoot
reb\_01 & Trolling & I find it exciting to poke fun at people \\
reb\_02 & Trolling & I find it exciting to poke fun at authority \\
reb\_03 & Trolling & I experience a thrill when disobeying authority \\
reb\_04 & Trolling & I experience an urge to disobey social rules \\
\end{longtable}

\subsection*{Coding}\label{coding}
\addcontentsline{toc}{subsection}{Coding}

This questionnaire follows our standard coding for extent based
measures, with To No Extent At All = 1, and To a Great Extent = 7.

\begin{longtable}[]{@{}
  >{\raggedright\arraybackslash}p{(\columnwidth - 12\tabcolsep) * \real{0.1429}}
  >{\raggedright\arraybackslash}p{(\columnwidth - 12\tabcolsep) * \real{0.1429}}
  >{\raggedright\arraybackslash}p{(\columnwidth - 12\tabcolsep) * \real{0.1429}}
  >{\raggedright\arraybackslash}p{(\columnwidth - 12\tabcolsep) * \real{0.1429}}
  >{\raggedright\arraybackslash}p{(\columnwidth - 12\tabcolsep) * \real{0.1429}}
  >{\raggedright\arraybackslash}p{(\columnwidth - 12\tabcolsep) * \real{0.1429}}
  >{\raggedright\arraybackslash}p{(\columnwidth - 12\tabcolsep) * \real{0.1429}}@{}}
\toprule\noalign{}
\begin{minipage}[b]{\linewidth}\raggedright
\textbf{1}
\end{minipage} & \begin{minipage}[b]{\linewidth}\raggedright
\textbf{2}
\end{minipage} & \begin{minipage}[b]{\linewidth}\raggedright
\textbf{3}
\end{minipage} & \begin{minipage}[b]{\linewidth}\raggedright
\textbf{4}
\end{minipage} & \begin{minipage}[b]{\linewidth}\raggedright
\textbf{5}
\end{minipage} & \begin{minipage}[b]{\linewidth}\raggedright
\textbf{6}
\end{minipage} & \begin{minipage}[b]{\linewidth}\raggedright
\textbf{7}
\end{minipage} \\
\midrule\noalign{}
\endhead
\bottomrule\noalign{}
\endlastfoot
To No Extent At All & To a Slight Extent & To a Mild Extent & To a
Moderate Extent & To an Appreciable Extent & To a Considerable Extent &
To a Great Extent \\
\end{longtable}

\texttt{No\ items\ are\ reverse\ coded}

\subsection{Scoring}\label{scoring}

The following variables are derived from this measure.

\begin{tabu} to \linewidth {>{\raggedright\arraybackslash}p{3cm}>{\raggedright\arraybackslash}p{3cm}>{\raggedright\arraybackslash}p{3cm}>{\raggedright\arraybackslash}p{3cm}>{\raggedright\arraybackslash}p{3cm}>{\raggedright\arraybackslash}p{3cm}}
\toprule
Variable Name & Variable Label & Description & Variable Type & Source (Section) & Definition\\
\midrule
Rebelliousness (trolling) & rebelliousnessTrolling & Rebelliousness trolling score & numeric & Rebelliousness & The mean of rb\_01 : rb\_04\\
\bottomrule
\end{tabu}

\chapter{CREDs}\label{creds}

\textbf{Cluster:} Socialisation

\section{Measure}\label{measure-1}

The measure is a further development of the CREDs measure used by Lanman
\& Buhrmester (2017).

\subsection*{Modifications}\label{modifications-1}
\addcontentsline{toc}{subsection}{Modifications}

The CRED measures that we use vary across the countries within our
survey, as such the following sections are subset by nation.
Specifically we have different questions for the historically Christian
nations (Brazil, Denmark, United Kingdom, United States), Japan, and
China, and we have nation specific examples in some cases.

In this round we added an additional question on perceived parental
sincerity- cred\_32.

\section{Implementation}\label{implementation-1}

\subsection*{Question wording}\label{question-wording-1}
\addcontentsline{toc}{subsection}{Question wording}

\begin{longtable}[]{@{}
  >{\raggedright\arraybackslash}p{(\columnwidth - 2\tabcolsep) * \real{0.0791}}
  >{\raggedright\arraybackslash}p{(\columnwidth - 2\tabcolsep) * \real{0.9209}}@{}}
\toprule\noalign{}
\begin{minipage}[b]{\linewidth}\raggedright
Questions
\end{minipage} & \begin{minipage}[b]{\linewidth}\raggedright
Question Text
\end{minipage} \\
\midrule\noalign{}
\endhead
\bottomrule\noalign{}
\endlastfoot
intro & The following questions ask about experiences during your
upbringing that relate to God. \\
cred\_01 - cred\_06 & These questions ask about your perceptions of your
primary caregiver or caregivers (i.e., parents or guardians). Please
answer each of the following according to your overall impression of
your caregiver(s) on the following scale: \\
cred\_07 & \emph{Stand alone question} \\
cred\_26 - cred\_31 & These questions ask about your perceptions of your
local area growing up (i.e., hometown, school). Please answer each of
the following according to your overall impression of your caregiver(s)
on the following scale: \\
\end{longtable}

\subsection*{Items}\label{items-1}
\addcontentsline{toc}{subsection}{Items}

The question items were as follows:

\begin{longtable}[]{@{}
  >{\raggedright\arraybackslash}p{(\columnwidth - 4\tabcolsep) * \real{0.0435}}
  >{\raggedright\arraybackslash}p{(\columnwidth - 4\tabcolsep) * \real{0.9000}}
  >{\raggedright\arraybackslash}p{(\columnwidth - 4\tabcolsep) * \real{0.0522}}@{}}
\toprule\noalign{}
\begin{minipage}[b]{\linewidth}\raggedright
Qlabel
\end{minipage} & \begin{minipage}[b]{\linewidth}\raggedright
question
\end{minipage} & \begin{minipage}[b]{\linewidth}\raggedright
response
\end{minipage} \\
\midrule\noalign{}
\endhead
\bottomrule\noalign{}
\endlastfoot
cred\_01 & To what extent did your caregivers visit places sacred to God
(e.g.~temple, church, synagogue, mosque)? & extent \\
cred\_02 & To what extent did your caregivers engage in
volunteer/charity work associated with their devotion to God.
(e.g.~Habitat for humanity, Salvation Army, Catholic Relief Services,
local religious charities)? & extent \\
cred\_03 & To what extent did your caregivers give financial donations
to Godly causes (e.g.~to churches,emples, mosques, charities, etc.)? &
extent \\
cred\_04 & To what extent did your caregivers make personal sacrifices
as part of their devotion to God (e.g.~fasting, abstaining from alcohol
and caffeine, giving away wealth, etc.)? & extent \\
cred\_05 & To what extent did your caregivers perform acts of devotion
to God in the home (e.g.~Bible readings, visible prayer, etc.)? &
extent \\
cred\_06 & To what extent do you think their devotion to God influenced
important decisions in their lives (e.g.~marriage, divorce, moving,
changing jobs, having children, etc.)? & extent \\
cred\_32 & How sincere did you feel your caregivers were in their
devotion to God? & sincerity \\
\end{longtable}

\begin{longtable}[]{@{}
  >{\raggedright\arraybackslash}p{(\columnwidth - 4\tabcolsep) * \real{0.0750}}
  >{\raggedright\arraybackslash}p{(\columnwidth - 4\tabcolsep) * \real{0.8600}}
  >{\raggedright\arraybackslash}p{(\columnwidth - 4\tabcolsep) * \real{0.0600}}@{}}
\toprule\noalign{}
\endhead
\bottomrule\noalign{}
\endlastfoot
cred\_07 & \begin{minipage}[t]{\linewidth}\raggedright
Did you attend or participate in any groups devoted to God{]} you were
growing up? (e.g.~churches, religious organisations, camps, etc.)\\

If yes, you will have the ability to answer for 2 different groups
below.\strut
\end{minipage} & yes/no \\
cred\_08 & Please name/describe the first group & \textbf{open} \\
cred\_09 & Did this group have leadership figures such as priests,
pastors, imams, head counsellors, rabbis, gurus, etc.? & yes/no \\
cred\_10 & How sincere did you feel the leaders of this group were in
their devotion to God? & sincerity \\
cred\_11 & To what extent did the leaders of this group make personal
sacrifices as part of their devotion to God? (e.g.~fasting, celibacy,
abstaining from alcohol and/or caffeine) & extent \\
cred\_12 & How sincere did you feel the members of this group were in
their devotion to God? & sincerity \\
cred\_13 & To what extent did members of this group engage in charitable
work together? & extent \\
cred\_14 & To what extent did members of this group express emotion
during group gatherings? & extent \\
cred\_15 & How often did members of this group appear to be in altered
states of consciousness such as speaking in tongues, trance, or
possession? & frequency \\
cred\_16 & To what extent would members of this group wear particular
clothes or clothing styles? & extent \\
cred\_17 - 25 & REPEAT CRED 08 - 16 & - \\
\end{longtable}

\begin{longtable}[]{@{}
  >{\raggedright\arraybackslash}p{(\columnwidth - 4\tabcolsep) * \real{0.0769}}
  >{\raggedright\arraybackslash}p{(\columnwidth - 4\tabcolsep) * \real{0.8154}}
  >{\raggedright\arraybackslash}p{(\columnwidth - 4\tabcolsep) * \real{0.1000}}@{}}
\toprule\noalign{}
\endhead
\bottomrule\noalign{}
\endlastfoot
cred\_26 & What percentage of people in your hometown/city do you think
believed in God? & percentage \\
cred\_27 & To what extent did believers in God in your hometown
demonstrate their beliefs in everyday life? & extent \\
cred\_28 & How sincere did you feel the believers in God in your
hometown were in their devotion to God? & sincerity \\
cred\_29 & What percentage of people in your high school do you think
believed in God? & percentage \\
cred\_30 & To what extent did the believers in God in your high school
demonstrate their beliefs in everyday life? & extent \\
cred\_31 & How sincere did you feel the believers in God in your high
school were in their devotion to God? & sincerity \\
\end{longtable}

\subsection{Country Specific
Variations:}\label{country-specific-variations}

\subsection*{Coding}\label{coding-1}
\addcontentsline{toc}{subsection}{Coding}

The above questions had a number of different response options, with no
reverse coding:

\subsubsection{Yes/No}\label{yesno}

\begin{longtable}[]{@{}ll@{}}
\toprule\noalign{}
0 & 1 \\
\midrule\noalign{}
\endhead
\bottomrule\noalign{}
\endlastfoot
No & Yes \\
\end{longtable}

\subsubsection{Frequency}\label{frequency}

\begin{longtable}[]{@{}
  >{\raggedright\arraybackslash}p{(\columnwidth - 12\tabcolsep) * \real{0.0891}}
  >{\raggedright\arraybackslash}p{(\columnwidth - 12\tabcolsep) * \real{0.1980}}
  >{\raggedright\arraybackslash}p{(\columnwidth - 12\tabcolsep) * \real{0.1485}}
  >{\raggedright\arraybackslash}p{(\columnwidth - 12\tabcolsep) * \real{0.1188}}
  >{\raggedright\arraybackslash}p{(\columnwidth - 12\tabcolsep) * \real{0.1287}}
  >{\raggedright\arraybackslash}p{(\columnwidth - 12\tabcolsep) * \real{0.1782}}
  >{\raggedright\arraybackslash}p{(\columnwidth - 12\tabcolsep) * \real{0.0891}}@{}}
\toprule\noalign{}
\endhead
\bottomrule\noalign{}
\endlastfoot
\textbf{1} & \textbf{2} & \textbf{3} & \textbf{4} & \textbf{5} &
\textbf{6} & \textbf{7} \\
Never & Very Infrequently & Infrequently & Sometimes & Frequently & Very
Frequently & Always \\
\end{longtable}

\subsubsection{Extent}\label{extent}

\begin{longtable}[]{@{}
  >{\raggedright\arraybackslash}p{(\columnwidth - 12\tabcolsep) * \real{0.2391}}
  >{\raggedright\arraybackslash}p{(\columnwidth - 12\tabcolsep) * \real{0.0978}}
  >{\raggedright\arraybackslash}p{(\columnwidth - 12\tabcolsep) * \real{0.0978}}
  >{\raggedright\arraybackslash}p{(\columnwidth - 12\tabcolsep) * \real{0.0978}}
  >{\raggedright\arraybackslash}p{(\columnwidth - 12\tabcolsep) * \real{0.0978}}
  >{\raggedright\arraybackslash}p{(\columnwidth - 12\tabcolsep) * \real{0.0978}}
  >{\raggedright\arraybackslash}p{(\columnwidth - 12\tabcolsep) * \real{0.2174}}@{}}
\toprule\noalign{}
\begin{minipage}[b]{\linewidth}\raggedright
\textbf{1}
\end{minipage} & \begin{minipage}[b]{\linewidth}\raggedright
\textbf{2}
\end{minipage} & \begin{minipage}[b]{\linewidth}\raggedright
\textbf{3}
\end{minipage} & \begin{minipage}[b]{\linewidth}\raggedright
\textbf{4}
\end{minipage} & \begin{minipage}[b]{\linewidth}\raggedright
\textbf{5}
\end{minipage} & \begin{minipage}[b]{\linewidth}\raggedright
\textbf{6}
\end{minipage} & \begin{minipage}[b]{\linewidth}\raggedright
\textbf{7}
\end{minipage} \\
\midrule\noalign{}
\endhead
\bottomrule\noalign{}
\endlastfoot
To No Extent At All & & & & & & To a Great Extent \\
\end{longtable}

\subsubsection{Sincerity}\label{sincerity}

\begin{longtable}[]{@{}
  >{\raggedright\arraybackslash}p{(\columnwidth - 12\tabcolsep) * \real{0.2035}}
  >{\raggedright\arraybackslash}p{(\columnwidth - 12\tabcolsep) * \real{0.1504}}
  >{\raggedright\arraybackslash}p{(\columnwidth - 12\tabcolsep) * \real{0.1062}}
  >{\raggedright\arraybackslash}p{(\columnwidth - 12\tabcolsep) * \real{0.0885}}
  >{\raggedright\arraybackslash}p{(\columnwidth - 12\tabcolsep) * \real{0.0885}}
  >{\raggedright\arraybackslash}p{(\columnwidth - 12\tabcolsep) * \real{0.1327}}
  >{\raggedright\arraybackslash}p{(\columnwidth - 12\tabcolsep) * \real{0.1858}}@{}}
\toprule\noalign{}
\begin{minipage}[b]{\linewidth}\raggedright
\textbf{1}
\end{minipage} & \begin{minipage}[b]{\linewidth}\raggedright
\textbf{2}
\end{minipage} & \begin{minipage}[b]{\linewidth}\raggedright
\textbf{3}
\end{minipage} & \begin{minipage}[b]{\linewidth}\raggedright
\textbf{4}
\end{minipage} & \begin{minipage}[b]{\linewidth}\raggedright
\textbf{5}
\end{minipage} & \begin{minipage}[b]{\linewidth}\raggedright
\textbf{6}
\end{minipage} & \begin{minipage}[b]{\linewidth}\raggedright
\textbf{7}
\end{minipage} \\
\midrule\noalign{}
\endhead
\bottomrule\noalign{}
\endlastfoot
Completely Insincere & Very Insincere & Insincere & Neutral & Sincere &
Very Sincere & Completely Sincere \\
\end{longtable}

\subsubsection{Percentage}\label{percentage}

\begin{longtable}[]{@{}l@{}}
\toprule\noalign{}
0 -100\% \\
\midrule\noalign{}
\endhead
\bottomrule\noalign{}
\endlastfoot
\end{longtable}

\subsection{Scoring}\label{scoring-1}

The following variables are derived from this measure:

\begin{tabu} to \linewidth {>{\raggedright\arraybackslash}p{3cm}>{\raggedright\arraybackslash}p{3cm}>{\raggedright\arraybackslash}p{3cm}>{\raggedright\arraybackslash}p{3cm}>{\raggedright\arraybackslash}p{3cm}>{\raggedright\arraybackslash}p{3cm}}
\toprule
Variable Name & Variable Label & Description & Variable Type & Source (Section) & Definition\\
\midrule
CRED Exposure & cred & CRED  exposure score & numeric & CREDs & sum of cred\_caregiver + cred\_group + cred\_hometown + cred\_highschool*\\
CRED (Caregiver) Exposure & credCaregiver & Caregiver CRED exposure score & numeric & CREDs & sum of cred\_01: cred\_06*\\
CRED (Group Leader) Exposure & credGroupLeader & Group leader CRED exposure score & numeric & CREDs & cred\_11 + cred\_20*\\
CRED (Group Member) Exposure & credGroupMember & Group member CRED exposure score & numeric & CREDs & sum of cred\_13, 14, 15, 16, 22, 23, 24, 25)*\\
CRED (Group) Exposure & credGroup & Group CRED exposure score & numeric & CREDs & sum of cred\_group\_ldr + cred\_group\_mbr*\\
\addlinespace
CRED (Hometown) Exposure & credHometown & Hometown CRED exposure score & ordinal & CREDs & cred\_28\\
CRED (High School) Exposure & credHighschool & Highschool CRED exposure score & ordinal & CREDs & cred\_30\\
Sincerity (Group Leader) & sincerityGroupLeader & Sincerity of group leader question response & ordinal & CREDs & mean of cred\_10: cred\_19*\\
Sincerity (Group Members) & sincerityGroupMember & Sincerity of group members question response & ordinal & CREDs & mean of cred\_12: cred\_21*\\
Sincerity (Parents) & sincerityParents & Sincerity of parents question response & ordinal & CREDs & cred\_32\\
\addlinespace
Sincerity (Hometown) & sincerityHometown & Sincerity of hometown question response & ordinal & CREDs & cred\_28\\
Sincerity (High school) & sincerityHighschool & Sincerity of highschool question response & ordinal & CREDs & cred\_31\\
Percentage Believers (Hometown) & believersHometown & Percentage of hometown believers response & numeric & CREDs & cred\_26\\
Percentage Believers (High School) & believersHighschool & Percentage of highschool believers response & numeric & CREDs & cred\_29\\
\bottomrule
\end{tabu}

\chapter{Normativity of Religion}\label{normativity-of-religion}

\section{Measure}\label{measure-2}

\subsection*{Modifications}\label{modifications-2}
\addcontentsline{toc}{subsection}{Modifications}

\section{Implementation}\label{implementation-2}

\subsection*{Question wording}\label{question-wording-2}
\addcontentsline{toc}{subsection}{Question wording}

Participants are presented with the following text:

\emph{People are a part of a number of groups that affect their lives,
such as nation-states, ethnic groups, companies, friend/peer groups, and
hobby groups (e.g.~sports/music fandoms). For these groups, being a
good, respected member often comes with expectations. Below, we will ask
you how important certain personal qualities are to being a good,
respected member of the groups to which you belong.}

Along with subscale specific text:

\begin{longtable}[]{@{}
  >{\raggedright\arraybackslash}p{(\columnwidth - 4\tabcolsep) * \real{0.1781}}
  >{\raggedright\arraybackslash}p{(\columnwidth - 4\tabcolsep) * \real{0.6438}}
  >{\raggedright\arraybackslash}p{(\columnwidth - 4\tabcolsep) * \real{0.1781}}@{}}
\toprule\noalign{}
\begin{minipage}[b]{\linewidth}\raggedright
\textbf{Subscale}
\end{minipage} & \begin{minipage}[b]{\linewidth}\raggedright
text
\end{minipage} & \begin{minipage}[b]{\linewidth}\raggedright
response
\end{minipage} \\
\midrule\noalign{}
\endhead
\bottomrule\noalign{}
\endlastfoot
Nation & The following questions will ask you about your nation. There
are no right or wrong answers and your responses remain anonymous. &
extent \\
Ethnic & The following questions will ask you about the ethnic group
with which you identify (e.g.~Black British, White (other), Han,
Japanese, Pacific Islander). There are no right or wrong answers and
your responses remain anonymous. & extent \\
Friends/peers & The following questions will ask you about your
friendship/ peer group. There are no right or wrong answers and your
responses remain anonymous. & extent \\
Choose own & The following questions will ask you about a hobby or
interest group that is important to you (e.g.~music, fashion, gaming
communities). There are no right or wrong answers and your responses
remain anonymous. & extent \\
\end{longtable}

\subsection*{Items}\label{items-2}
\addcontentsline{toc}{subsection}{Items}

The question items were as follows, where {[}COUNTRY{]} is the name of
the country:

\begin{longtable}[]{@{}
  >{\raggedright\arraybackslash}p{(\columnwidth - 6\tabcolsep) * \real{0.1644}}
  >{\raggedright\arraybackslash}p{(\columnwidth - 6\tabcolsep) * \real{0.1644}}
  >{\raggedright\arraybackslash}p{(\columnwidth - 6\tabcolsep) * \real{0.5068}}
  >{\raggedright\arraybackslash}p{(\columnwidth - 6\tabcolsep) * \real{0.1644}}@{}}
\toprule\noalign{}
\begin{minipage}[b]{\linewidth}\raggedright
Qlabel
\end{minipage} & \begin{minipage}[b]{\linewidth}\raggedright
\textbf{Subscale}
\end{minipage} & \begin{minipage}[b]{\linewidth}\raggedright
question
\end{minipage} & \begin{minipage}[b]{\linewidth}\raggedright
response
\end{minipage} \\
\midrule\noalign{}
\endhead
\bottomrule\noalign{}
\endlastfoot
norm\_01 & Nation & To what extent is it expected for a citizen of
{[}COUNTRY{]} to believe in God? & extent \\
norm\_02 & Nation & To what extent is it expected for a citizen of
{[}COUNTRY{]} to perform rituals honouring God? & extent \\
norm\_03 & Nation & To what extent is it expected for a citizen of
{[}COUNTRY{]} to belong to a religion? & extent \\
norm\_04 & Ethnic & To what extent is it expected of people in your
ethnic group to believe in God? & extent \\
norm\_05 & Ethnic & To what extent is it expected of people in your
ethnic group to perform rituals honouring God? & extent \\
norm\_06 & Ethnic & To what extent is it expected of people in your
ethnic group to belong to a religion? & extent \\
norm\_07 & Friends/peers & To what extent is it expected amongst your
friends to believe in God? & extent \\
norm\_08 & Friends/peers & To what extent is it expected amongst your
friends to perform rituals honouring God? & extent \\
norm\_09 & Friends/peers & To what extent is it expected amongst your
friends to belong to a religion? & extent \\
norm\_10 & Choose own & To what extent is it expected for members of the
hobby or interest group most important to you to believe in God? &
extent \\
norm\_11 & Choose own & To what extent is it expected for members of the
hobby or interest group most important to you to perform rituals for
God? & extent \\
norm\_12 & Choose own & To what extent is it expected for members of the
hobby or interest group most important to you to belong to a religion? &
extent \\
\end{longtable}

\subsection*{Coding}\label{coding-2}
\addcontentsline{toc}{subsection}{Coding}

This questionnaire follows our standard coding for extent based
measures, with To No Extent At All = 1, and To a Great Extent = 7.

\begin{longtable}[]{@{}
  >{\raggedright\arraybackslash}p{(\columnwidth - 12\tabcolsep) * \real{0.1429}}
  >{\raggedright\arraybackslash}p{(\columnwidth - 12\tabcolsep) * \real{0.1429}}
  >{\raggedright\arraybackslash}p{(\columnwidth - 12\tabcolsep) * \real{0.1429}}
  >{\raggedright\arraybackslash}p{(\columnwidth - 12\tabcolsep) * \real{0.1429}}
  >{\raggedright\arraybackslash}p{(\columnwidth - 12\tabcolsep) * \real{0.1429}}
  >{\raggedright\arraybackslash}p{(\columnwidth - 12\tabcolsep) * \real{0.1429}}
  >{\raggedright\arraybackslash}p{(\columnwidth - 12\tabcolsep) * \real{0.1429}}@{}}
\toprule\noalign{}
\begin{minipage}[b]{\linewidth}\raggedright
\textbf{1}
\end{minipage} & \begin{minipage}[b]{\linewidth}\raggedright
\textbf{2}
\end{minipage} & \begin{minipage}[b]{\linewidth}\raggedright
\textbf{3}
\end{minipage} & \begin{minipage}[b]{\linewidth}\raggedright
\textbf{4}
\end{minipage} & \begin{minipage}[b]{\linewidth}\raggedright
\textbf{5}
\end{minipage} & \begin{minipage}[b]{\linewidth}\raggedright
\textbf{6}
\end{minipage} & \begin{minipage}[b]{\linewidth}\raggedright
\textbf{7}
\end{minipage} \\
\midrule\noalign{}
\endhead
\bottomrule\noalign{}
\endlastfoot
To No Extent At All & To a Slight Extent & To a Mild Extent & To a
Moderate Extent & To an Appreciable Extent & To a Considerable Extent &
To a Great Extent \\
\end{longtable}

\texttt{No\ items\ are\ reverse\ coded}

\subsection{Scoring}\label{scoring-2}

\begin{tabu} to \linewidth {>{\raggedright\arraybackslash}p{3cm}>{\raggedright\arraybackslash}p{3cm}>{\raggedright\arraybackslash}p{3cm}>{\raggedright\arraybackslash}p{3cm}>{\raggedright\arraybackslash}p{3cm}>{\raggedright\arraybackslash}p{3cm}}
\toprule
Variable Name & Variable Label & Description & Variable Type & Source (Section) & Definition\\
\midrule
Normativity of Religion (all) & normativity & Normativity of religion score & numeric & Normativity of Religion & mean of norm\_01: norm\_12\\
Normativity of Religion (Nation) & nomrativityNation & Nation normativity of religion score & ordinal & Normativity of Religion & mean of norm\_01: norm\_03\\
Normativity of Religion (Ethnicity) & normativityEthnicity & Ethnicity normativity of religion score & ordinal & Normativity of Religion & mean of norm\_04: norm\_06\\
Normativity of Religion (Peers) & normativityPeers & Peer normativity of religion score & ordinal & Normativity of Religion & mean of norm\_07: norm\_09\\
Normativity of Religion (Group) & normativityGroup & Group normativity of religion score & ordinal & Normativity of Religion & mean of norm\_10: norm\_12\\
\bottomrule
\end{tabu}

\chapter{Collectivism}\label{collectivism}

\section{Measure}\label{measure-3}

The collectivism measure used was that of Triandis (1994), available in
Gelfand \& Realo (1999).

\subsection*{Modifications}\label{modifications-3}
\addcontentsline{toc}{subsection}{Modifications}

For consistency with our other measures we used a 7-point likert scale
for agreement, where the original measure used a 9-point scale.

\section{Implementation}\label{implementation-3}

\subsection*{Question wording}\label{question-wording-3}
\addcontentsline{toc}{subsection}{Question wording}

Participants read the following text:

\emph{We want to know if you agree or disagree with the following
statements. The statements sometimes refer to your `group,' which refers
to your group of friends or any other group that you are involved in.
Read each one carefully. Indicate your agreement or disagreement with
the statement by using the following scale:}

\subsection*{Items}\label{items-3}
\addcontentsline{toc}{subsection}{Items}

\begin{longtable}[]{@{}
  >{\raggedright\arraybackslash}p{(\columnwidth - 2\tabcolsep) * \real{0.1250}}
  >{\raggedright\arraybackslash}p{(\columnwidth - 2\tabcolsep) * \real{0.8750}}@{}}
\toprule\noalign{}
\begin{minipage}[b]{\linewidth}\raggedright
Qlabel
\end{minipage} & \begin{minipage}[b]{\linewidth}\raggedright
question
\end{minipage} \\
\midrule\noalign{}
\endhead
\bottomrule\noalign{}
\endlastfoot
ic\_01 & It is important for me to maintain harmony within my group \\
ic\_02 & I would sacrifice an activity that I enjoy very much if my
family did not approve of it. \\
ic\_03 & Children should be taught to place duty before pleasure. \\
ic\_04 & My happiness depends very much on the happiness of those around

me. \\
ic\_05 & The well-being of my group is a very important concern for
me \\
ic\_06 & I really like to cooperate with others \\
ic\_07 & I usually sacrifice my self-interest for the benefit of my
group \\
ic\_08 & Before making a decision, I like to consult with many
others. \\
ic\_09 & Children should feel honored if their parents receive a
distinguished

award \\
ic\_10 & If any of my relatives were in financial difficulty, I would
help them

even if it made my life difficult. \\
ic\_11 & If a member of my group gets a prize, I would feel proud. \\
ic\_12 & Sharing little things with my group makes me very happy \\
ic\_13 & I feel we should keep our aging parents with us at home \\
ic\_14 & To me, pleasure is spending time with others \\
ic\_15 & I hate to disagree with others in my group \\
ic\_16 & I would do what would please my family, even if I detested the
activity \\
\end{longtable}

\subsection*{Coding}\label{coding-3}
\addcontentsline{toc}{subsection}{Coding}

This questionnaire used our standard response scale for agreement.

\begin{longtable}[]{@{}
  >{\raggedright\arraybackslash}p{(\columnwidth - 12\tabcolsep) * \real{0.1342}}
  >{\raggedright\arraybackslash}p{(\columnwidth - 12\tabcolsep) * \real{0.1477}}
  >{\raggedright\arraybackslash}p{(\columnwidth - 12\tabcolsep) * \real{0.1342}}
  >{\raggedright\arraybackslash}p{(\columnwidth - 12\tabcolsep) * \real{0.1946}}
  >{\raggedright\arraybackslash}p{(\columnwidth - 12\tabcolsep) * \real{0.1141}}
  >{\raggedright\arraybackslash}p{(\columnwidth - 12\tabcolsep) * \real{0.1275}}
  >{\raggedright\arraybackslash}p{(\columnwidth - 12\tabcolsep) * \real{0.1141}}@{}}
\toprule\noalign{}
\endhead
\bottomrule\noalign{}
\endlastfoot
\textbf{1} & \textbf{2} & \textbf{3} & \textbf{4} & \textbf{5} &
\textbf{6} & \textbf{7} \\
strongly disagree & moderately disagree & slightly disagree & neither
agree nor disagree & slightly agree & moderately agree & strongly
agree \\
\end{longtable}

\texttt{No\ items\ are\ reverse\ coded}

\subsection*{Scoring}\label{scoring-3}
\addcontentsline{toc}{subsection}{Scoring}

The following variables were derived from this measure

\begin{tabu} to \linewidth {>{\raggedright\arraybackslash}p{3cm}>{\raggedright\arraybackslash}p{3cm}>{\raggedright\arraybackslash}p{3cm}>{\raggedright\arraybackslash}p{3cm}>{\raggedright\arraybackslash}p{3cm}>{\raggedright}X}
\toprule
Variable Name & Variable Label & Description & Variable Type & Source (Section) & Definition\\
\midrule
Collectivism & collectivism & Individualism score & numeric & Individualism \& Collectivism & The mean of ic\_01 : ic\_16\\
\bottomrule
\end{tabu}

\section{}\label{section}

\chapter{Existential Insecurity}\label{existential-insecurity}

\textbf{Cluster:} Motivational

\section{Measure}\label{measure-4}

The initial existential insecurity measures were derived from those used
by Baimel et al. (2022) and by Willard \& Cingl (2017).

\subsection*{Modifications}\label{modifications-4}
\addcontentsline{toc}{subsection}{Modifications}

The food security measure from Baimel et al. (2022) was combined with
financial and physical security from Willard \& Cingl (2017), these were
asked for `recently' and `for the forthcoming months'.

In this wave we only included questions pertaining to childhood.

\section{Implementation}\label{implementation-4}

\subsection*{Question wording}\label{question-wording-4}
\addcontentsline{toc}{subsection}{Question wording}

The question wording was as follows:

\begin{longtable}[]{@{}
  >{\raggedright\arraybackslash}p{(\columnwidth - 2\tabcolsep) * \real{0.2361}}
  >{\raggedright\arraybackslash}p{(\columnwidth - 2\tabcolsep) * \real{0.7639}}@{}}
\toprule\noalign{}
\begin{minipage}[b]{\linewidth}\raggedright
Questions
\end{minipage} & \begin{minipage}[b]{\linewidth}\raggedright
Question Text
\end{minipage} \\
\midrule\noalign{}
\endhead
\bottomrule\noalign{}
\endlastfoot
es\_15 - es\_21 & \emph{When growing up did you:} \\
\end{longtable}

\subsection*{Items}\label{items-4}
\addcontentsline{toc}{subsection}{Items}

The question items were as follows:

\textbf{Existential Security}

\begin{longtable}[]{@{}
  >{\raggedright\arraybackslash}p{(\columnwidth - 2\tabcolsep) * \real{0.1806}}
  >{\raggedright\arraybackslash}p{(\columnwidth - 2\tabcolsep) * \real{0.8194}}@{}}
\toprule\noalign{}
\begin{minipage}[b]{\linewidth}\raggedright
Qlabel
\end{minipage} & \begin{minipage}[b]{\linewidth}\raggedright
question
\end{minipage} \\
\midrule\noalign{}
\endhead
\bottomrule\noalign{}
\endlastfoot
es\_15 & worry that your household may not able to buy or produce enough
food to eat? \\
es\_16 & feel that your household may not be able to afford to buy items
you need \\
es\_17 & worry that your household may not have enough money \\
es\_18 & worry about members of your household losing their job \\
es\_19 & feel you will be safe walking alone in your local area after
dark \\
es\_20 & worry about being burgled \\
es\_21 & worry about being a victim of violent crime \\
\end{longtable}

\subsection*{Coding}\label{coding-4}
\addcontentsline{toc}{subsection}{Coding}

Items on the existential security subscale use the following frequency
based Likert scale:

\begin{longtable}[]{@{}
  >{\raggedright\arraybackslash}p{(\columnwidth - 12\tabcolsep) * \real{0.1429}}
  >{\raggedright\arraybackslash}p{(\columnwidth - 12\tabcolsep) * \real{0.1429}}
  >{\raggedright\arraybackslash}p{(\columnwidth - 12\tabcolsep) * \real{0.1429}}
  >{\raggedright\arraybackslash}p{(\columnwidth - 12\tabcolsep) * \real{0.1429}}
  >{\raggedright\arraybackslash}p{(\columnwidth - 12\tabcolsep) * \real{0.1429}}
  >{\raggedright\arraybackslash}p{(\columnwidth - 12\tabcolsep) * \real{0.1429}}
  >{\raggedright\arraybackslash}p{(\columnwidth - 12\tabcolsep) * \real{0.1429}}@{}}
\toprule\noalign{}
\endhead
\bottomrule\noalign{}
\endlastfoot
\textbf{1} & \textbf{2} & \textbf{3} & \textbf{4} & \textbf{5} &
\textbf{6} & \textbf{7} \\
never & very infrequently & infrequently & sometimes & frequently & very
frequently & always \\
\end{longtable}

\texttt{es\_19\ is\ reverse\ coded}

\subsection{Scoring}\label{scoring-4}

The following variables are derived form this measure:

\begin{tabu} to \linewidth {>{\raggedright\arraybackslash}p{3cm}>{\raggedright\arraybackslash}p{3cm}>{\raggedright\arraybackslash}p{3cm}>{\raggedright\arraybackslash}p{3cm}>{\raggedright\arraybackslash}p{3cm}>{\raggedright\arraybackslash}p{3cm}}
\toprule
Variable Name & Variable Label & Description & Variable Type & Source (Section) & Definition\\
\midrule
Existential Security (upbringing) & ex\_sec\_upb & Upbringing existential security score & numeric & Existential Security & mean es\_15: es\_21\\
\bottomrule
\end{tabu}

\chapter{Non-theistic Socialisation}\label{non-theistic-socialisation}

\textbf{Cluster:} Socialisation

\section{Measure}\label{measure-5}

\subsection*{Modifications}\label{modifications-5}
\addcontentsline{toc}{subsection}{Modifications}

\section{Implementation}\label{implementation-5}

The Non-theistic Socialisation measure that we use vary across the
countries within our survey, as such the following sections are subset
by nation. Specifically we have different questions for the historically
Christian nations (Brazil, Denmark, United Kingdom, United States), and
for Japan and China.

\subsection*{Question wording}\label{question-wording-5}
\addcontentsline{toc}{subsection}{Question wording}

\begin{longtable}[]{@{}
  >{\raggedright\arraybackslash}p{(\columnwidth - 2\tabcolsep) * \real{0.2083}}
  >{\raggedright\arraybackslash}p{(\columnwidth - 2\tabcolsep) * \real{0.7917}}@{}}
\toprule\noalign{}
\begin{minipage}[b]{\linewidth}\raggedright
Questions
\end{minipage} & \begin{minipage}[b]{\linewidth}\raggedright
Question Text
\end{minipage} \\
\midrule\noalign{}
\endhead
\bottomrule\noalign{}
\endlastfoot
intro & To what extent did important people in your upbringing speak
against the idea that God exists? That is, to what extent, adding it all
up, did the important people in your life -- such as your parents,
teachers, and church officials (if any) -- do the things listed below as
you were growing up? \\
\end{longtable}

\subsection*{Items}\label{items-5}
\addcontentsline{toc}{subsection}{Items}

The items were as follows:

\begin{longtable}[]{@{}
  >{\raggedright\arraybackslash}p{(\columnwidth - 4\tabcolsep) * \real{0.2055}}
  >{\raggedright\arraybackslash}p{(\columnwidth - 4\tabcolsep) * \real{0.5890}}
  >{\raggedright\arraybackslash}p{(\columnwidth - 4\tabcolsep) * \real{0.2055}}@{}}
\toprule\noalign{}
\begin{minipage}[b]{\linewidth}\raggedright
Qlabel
\end{minipage} & \begin{minipage}[b]{\linewidth}\raggedright
question
\end{minipage} & \begin{minipage}[b]{\linewidth}\raggedright
response
\end{minipage} \\
\midrule\noalign{}
\endhead
\bottomrule\noalign{}
\endlastfoot
nts\_01 & State that God does not exist? & extent \\
nts\_02 & State that prayers to God do not work? & extent \\
nts\_03 & Mock or joke about people's religious belief? & extent \\
nts\_04 & State that belief in God is not necessary for morality? &
extent \\
nts\_05 & State that one should not be religious? & extent \\
nts\_06 & State that there are no good reasons to believe in God? &
extent \\
nts\_07 & State that religious belief does more harm than good in the
world? & extent \\
nts\_08 & State that religious practices are irrational? & extent \\
\end{longtable}

\subsection*{Coding}\label{coding-5}
\addcontentsline{toc}{subsection}{Coding}

This measure uses our standard response scale for extent.

\subsubsection{Extent}\label{extent-1}

\begin{longtable}[]{@{}
  >{\raggedright\arraybackslash}p{(\columnwidth - 12\tabcolsep) * \real{0.2800}}
  >{\raggedright\arraybackslash}p{(\columnwidth - 12\tabcolsep) * \real{0.0933}}
  >{\raggedright\arraybackslash}p{(\columnwidth - 12\tabcolsep) * \real{0.0933}}
  >{\raggedright\arraybackslash}p{(\columnwidth - 12\tabcolsep) * \real{0.0933}}
  >{\raggedright\arraybackslash}p{(\columnwidth - 12\tabcolsep) * \real{0.0933}}
  >{\raggedright\arraybackslash}p{(\columnwidth - 12\tabcolsep) * \real{0.0933}}
  >{\raggedright\arraybackslash}p{(\columnwidth - 12\tabcolsep) * \real{0.2533}}@{}}
\toprule\noalign{}
\endhead
\bottomrule\noalign{}
\endlastfoot
\textbf{1} & \textbf{2} & \textbf{3} & \textbf{4} & \textbf{5} &
\textbf{6} & \textbf{7} \\
To No Extent At All & & & & & & To a Great Extent \\
\end{longtable}

\texttt{No\ items\ were\ reverse\ coded}

\subsection{Scoring}\label{scoring-5}

The following variables were derived from this measure:

\begin{tabu} to \linewidth {>{\raggedright\arraybackslash}p{3cm}>{\raggedright\arraybackslash}p{3cm}>{\raggedright\arraybackslash}p{3cm}>{\raggedright\arraybackslash}p{3cm}>{\raggedright\arraybackslash}p{3cm}>{\raggedright\arraybackslash}p{3cm}}
\toprule
Variable Name & Variable Label & Description & Variable Type & Source (Section) & Definition\\
\midrule
Non-Theistic Socialisation & nontheisticSocialisation & Non-theistic socialisation score & numeric & Non Theistic Socialisation & mean of nts\_01: nts\_08\\
\bottomrule
\end{tabu}

\section{}\label{section-1}

\chapter{Religious Emphasis}\label{religious-emphasis}

\section{Measure}\label{measure-6}

\section{Implementation}\label{implementation-6}

The Religious Emphasis measure that we use vary across the countries
within our survey. Specifically we have different question wording for
the historically Christian nations (Brazil, Denmark, United Kingdom,
United States), Japan, and China.

\subsection*{Question wording}\label{question-wording-6}
\addcontentsline{toc}{subsection}{Question wording}

\begin{longtable}[]{@{}
  >{\raggedright\arraybackslash}p{(\columnwidth - 2\tabcolsep) * \real{0.0407}}
  >{\raggedright\arraybackslash}p{(\columnwidth - 2\tabcolsep) * \real{0.9593}}@{}}
\toprule\noalign{}
\begin{minipage}[b]{\linewidth}\raggedright
Questions
\end{minipage} & \begin{minipage}[b]{\linewidth}\raggedright
Question Text
\end{minipage} \\
\midrule\noalign{}
\endhead
\bottomrule\noalign{}
\endlastfoot
intro & To what extent did you have an upbringing devoted to God? That
is, to what extent, adding it all up, did the important people in your
life -- such as your parents, teachers, and church officials (if any) --
do the things listed below as you were growing up? \\
\end{longtable}

\section{Items}\label{items-6}

The items were as follows:

\begin{longtable}[]{@{}
  >{\raggedright\arraybackslash}p{(\columnwidth - 4\tabcolsep) * \real{0.0519}}
  >{\raggedright\arraybackslash}p{(\columnwidth - 4\tabcolsep) * \real{0.8831}}
  >{\raggedright\arraybackslash}p{(\columnwidth - 4\tabcolsep) * \real{0.0649}}@{}}
\toprule\noalign{}
\begin{minipage}[b]{\linewidth}\raggedright
Qlabel
\end{minipage} & \begin{minipage}[b]{\linewidth}\raggedright
question
\end{minipage} & \begin{minipage}[b]{\linewidth}\raggedright
response
\end{minipage} \\
\midrule\noalign{}
\endhead
\bottomrule\noalign{}
\endlastfoot
re\_01 & Review the teachings of Christianity, Judaism, Islam, Hinduism
or another similar tradition at home? & extent \\
re\_02 & Emphasize that you should read scripture or books associated
with Christianity, Judaism, Islam, Hinduism or another similar
tradition? & extent \\
re\_03 & Discuss moral do's and don'ts in relation to God? & extent \\
re\_04 & Tell stories about God? & extent \\
re\_05 & Talk about the afterlife? & extent \\
re\_06 & Emphasise that spiritual matters were of high importance? &
extent \\
re\_07 & Relate stories about God to contemporary life? & extent \\
re\_08 & Teach you to obey the wishes of God? & extent \\
re\_09 & Teach you to obey the persons who act as representatives of God
(e.g.~priests, ministers, rabbis, imams, etc.)? & extent \\
\end{longtable}

\subsection*{Coding}\label{coding-6}
\addcontentsline{toc}{subsection}{Coding}

This measure used our standard response scale for extent.

\begin{longtable}[]{@{}
  >{\raggedright\arraybackslash}p{(\columnwidth - 12\tabcolsep) * \real{0.1382}}
  >{\raggedright\arraybackslash}p{(\columnwidth - 12\tabcolsep) * \real{0.1316}}
  >{\raggedright\arraybackslash}p{(\columnwidth - 12\tabcolsep) * \real{0.1184}}
  >{\raggedright\arraybackslash}p{(\columnwidth - 12\tabcolsep) * \real{0.1447}}
  >{\raggedright\arraybackslash}p{(\columnwidth - 12\tabcolsep) * \real{0.1711}}
  >{\raggedright\arraybackslash}p{(\columnwidth - 12\tabcolsep) * \real{0.1711}}
  >{\raggedright\arraybackslash}p{(\columnwidth - 12\tabcolsep) * \real{0.1250}}@{}}
\toprule\noalign{}
\endhead
\bottomrule\noalign{}
\endlastfoot
\textbf{1} & \textbf{2} & \textbf{3} & \textbf{4} & \textbf{5} &
\textbf{6} & \textbf{7} \\
To No Extent At All & To a Slight Extent & To a Mild Extent & To a
Moderate Extent & To an Appreciable Extent & To a Considerable Extent &
To a Great Extent \\
\end{longtable}

\texttt{No\ items\ are\ reverse\ coded}

\subsection{Scoring}\label{scoring-6}

The following variables were derived from this measure.

\begin{tabu} to \linewidth {>{\raggedright\arraybackslash}p{3cm}>{\raggedright\arraybackslash}p{3cm}>{\raggedright\arraybackslash}p{3cm}>{\raggedright\arraybackslash}p{3cm}>{\raggedright\arraybackslash}p{3cm}>{\raggedright\arraybackslash}p{3cm}}
\toprule
Variable Name & Variable Label & Description & Variable Type & Source (Section) & Definition\\
\midrule
Religious Emphasis & religiousEmphasis & Religious emphasis score & numeric & Religious Emphasis & mean of re\_01: re\_09\\
\bottomrule
\end{tabu}

\chapter{Social Desirability}\label{social-desirability}

\textbf{Cluster:} Motivational

\section{Measure}\label{measure-7}

To measure social desirability we use is the impression management
subscale of the balanced Inventory of Desirable Responding Short Form
(BIDR-16) from Hart, Ritchie, Hepper, \& Gebauer (2015) . This itself is
an abbreviated version of the larger 40-item BIDR measure which can be
found
\href{https://sjdm.org/dmidi/Balanced_Inventory_of_Desirable_Responding.html}{here}.

\subsection*{Modifications}\label{modifications-6}
\addcontentsline{toc}{subsection}{Modifications}

The BIDR-16 used a truth based Likert scale. As with other measures we
have standardized this to be a 7-point scale with anchors and response
items consistent with the other truth-based Likert scales we used in the
project.

The original BIDR measure asked participants to write a numbered
response indicating the degree to which each statement is true. For
consistency with other measures used, and the method of administration,
we ask participants to respond by selecting their response.

\section{Implementation}\label{implementation-7}

\subsection*{Question wording}\label{question-wording-7}
\addcontentsline{toc}{subsection}{Question wording}

Participants read the following text:

\emph{Please read each of the following statements carefully and say to
what degree they are true or untrue for you. There are no right or wrong
answers and your responses remain anonymous.}

\subsection*{Items}\label{items-7}
\addcontentsline{toc}{subsection}{Items}

\begin{longtable}[]{@{}
  >{\raggedright\arraybackslash}p{(\columnwidth - 2\tabcolsep) * \real{0.2676}}
  >{\raggedright\arraybackslash}p{(\columnwidth - 2\tabcolsep) * \real{0.7324}}@{}}
\toprule\noalign{}
\begin{minipage}[b]{\linewidth}\raggedright
Qlabel
\end{minipage} & \begin{minipage}[b]{\linewidth}\raggedright
question
\end{minipage} \\
\midrule\noalign{}
\endhead
\bottomrule\noalign{}
\endlastfoot
sd\_01 & I sometimes tell lies if I have to \\
sd\_02 & I never cover up my mistakes \\
sd\_03 & There have been occasions when I have taken advantage of
someone \\
sd\_04 & I sometimes try to get even rather than forgive and forget \\
sd\_05 & I have said something bad about a friend behind his/her back \\
sd\_06 & When I hear people talking privately, I avoid listening \\
sd\_07 & I never take things that don't belong to me \\
sd\_08 & I don't gossip about other people's business \\
\end{longtable}

\subsection*{Coding}\label{coding-7}
\addcontentsline{toc}{subsection}{Coding}

This questionnaire follows our standard coding for agreement based
measures, with absolutely untrue = 1, and absolutely true = 7.

\begin{longtable}[]{@{}
  >{\raggedright\arraybackslash}p{(\columnwidth - 12\tabcolsep) * \real{0.1429}}
  >{\raggedright\arraybackslash}p{(\columnwidth - 12\tabcolsep) * \real{0.1429}}
  >{\raggedright\arraybackslash}p{(\columnwidth - 12\tabcolsep) * \real{0.1429}}
  >{\raggedright\arraybackslash}p{(\columnwidth - 12\tabcolsep) * \real{0.1429}}
  >{\raggedright\arraybackslash}p{(\columnwidth - 12\tabcolsep) * \real{0.1429}}
  >{\raggedright\arraybackslash}p{(\columnwidth - 12\tabcolsep) * \real{0.1429}}
  >{\raggedright\arraybackslash}p{(\columnwidth - 12\tabcolsep) * \real{0.1429}}@{}}
\toprule\noalign{}
\endhead
\bottomrule\noalign{}
\endlastfoot
\textbf{1} & \textbf{2} & \textbf{3} & \textbf{4} & \textbf{5} &
\textbf{6} & \textbf{7} \\
absolutely untrue & mostly untrue & somewhat untrue & can't say true or
false & somewhat true & mostly true & absolutely true \\
\end{longtable}

\texttt{Items\ sd\_01,\ sd\_03,\ sd\_04,\ and\ sd\_05\ are\ reverse\ coded}

\subsection{Scoring}\label{scoring-7}

The following variables are derived from this measure:

\chapter{Meaning in Life}\label{meaning-in-life}

\textbf{Cluster:} Motivational

\section{Measure}\label{measure-8}

To examine need for meaning we used The Meaning in Life Questionnaire
(MLQ) by Steger, Frazier, Oishi, \& Kaler (2006). THe original scale has
a two factor structure, with the subscales noted below.

\subsection*{Modifications}\label{modifications-7}
\addcontentsline{toc}{subsection}{Modifications}

The truth-based Likert scale in the MLQ is the standard 7-point truth
scale we used throughout the project so this was not altered.

We used the original question wording for the MLQ.

\section{Implementation}\label{implementation-8}

\subsection*{Question wording}\label{question-wording-8}
\addcontentsline{toc}{subsection}{Question wording}

Participants read the following text:

\emph{Please read each of the following statements carefully and rate
how strongly you agree or disagree. There are no right or wrong answers
and your responses remain anonymous.}

\subsection{Items}\label{items-8}

\textbf{Meaning in Life Questionnaire}

\begin{longtable}[]{@{}
  >{\raggedright\arraybackslash}p{(\columnwidth - 4\tabcolsep) * \real{0.1944}}
  >{\raggedright\arraybackslash}p{(\columnwidth - 4\tabcolsep) * \real{0.1944}}
  >{\raggedright\arraybackslash}p{(\columnwidth - 4\tabcolsep) * \real{0.6111}}@{}}
\toprule\noalign{}
\begin{minipage}[b]{\linewidth}\raggedright
Qlabel
\end{minipage} & \begin{minipage}[b]{\linewidth}\raggedright
Subscale
\end{minipage} & \begin{minipage}[b]{\linewidth}\raggedright
question
\end{minipage} \\
\midrule\noalign{}
\endhead
\bottomrule\noalign{}
\endlastfoot
mlq\_01 & presence & I understand my life's meaning. \\
mlq\_02 & search & I am looking for something that makes my life feel
meaningful. \\
mlq\_03 & search & I am always looking to find my life's purpose. \\
mlq\_04 & presence & My life has a clear sense of purpose \\
mlq\_05 & presence & I have a good sense of what makes my life
meaningful. \\
mlq\_06 & presence & I have discovered a satisfying life purpose. \\
mlq\_07 & search & I am always searching for something that makes my
life feel significant. \\
mlq\_08 & search & I am seeking a purpose or mission for my life. \\
mlq\_09 & presence & My life has no clear purpose. \\
mlq\_10 & search & I am searching for meaning in my life. \\
\end{longtable}

\section{Coding}\label{coding-8}

\textbf{The Meaning in Life Questionnaire (mlq\_ items) used our
standard response scale for truth:}

\begin{longtable}[]{@{}
  >{\raggedright\arraybackslash}p{(\columnwidth - 12\tabcolsep) * \real{0.1429}}
  >{\raggedright\arraybackslash}p{(\columnwidth - 12\tabcolsep) * \real{0.1429}}
  >{\raggedright\arraybackslash}p{(\columnwidth - 12\tabcolsep) * \real{0.1429}}
  >{\raggedright\arraybackslash}p{(\columnwidth - 12\tabcolsep) * \real{0.1429}}
  >{\raggedright\arraybackslash}p{(\columnwidth - 12\tabcolsep) * \real{0.1429}}
  >{\raggedright\arraybackslash}p{(\columnwidth - 12\tabcolsep) * \real{0.1429}}
  >{\raggedright\arraybackslash}p{(\columnwidth - 12\tabcolsep) * \real{0.1429}}@{}}
\toprule\noalign{}
\endhead
\bottomrule\noalign{}
\endlastfoot
\textbf{1} & \textbf{2} & \textbf{3} & \textbf{4} & \textbf{5} &
\textbf{6} & \textbf{7} \\
absolutely untrue & mostly untrue & somewhat untrue & can't say true or
false & somewhat true & mostly true & absolutely true \\
\end{longtable}

\texttt{Item\ mlq\_09\ is\ reverse\ coded}

\section{Scoring}\label{scoring-8}

\begin{tabu} to \linewidth {>{\raggedright\arraybackslash}p{3cm}>{\raggedright\arraybackslash}p{3cm}>{\raggedright\arraybackslash}p{3cm}>{\raggedright\arraybackslash}p{3cm}>{\raggedright\arraybackslash}p{3cm}>{\raggedright\arraybackslash}p{3cm}}
\toprule
Variable Name & Variable Label & Description & Variable Type & Source (Section) & Definition\\
\midrule
Meaning in Life & meaningLife & Meaning in life score & numeric & Need for Meaning & mean mlq\_01: mlq\_10\\
\bottomrule
\end{tabu}

\part{Survey 2}

\chapter{Anthropomorphism}\label{anthropomorphism}

\section{Measure}\label{measure-9}

\subsection*{Modifications}\label{modifications-8}
\addcontentsline{toc}{subsection}{Modifications}

The original measure used by Neave, Jackson, Saxton, \& Hönekopp (2015)
questions pertaining to childhood and adulthood experiences. For the
sake of brevity only questions pertaining to adulthood were included.

For consistency with the other agreement based measures used in the
project the anchors and response items were altered to be a 7-point
Likert scale.

\section{Implementation}\label{implementation-9}

\subsection*{Question wording}\label{question-wording-9}
\addcontentsline{toc}{subsection}{Question wording}

Participants read the following text:

\emph{Please read each of the following statements carefully and rate
how strongly you agree or disagree. There are no right or wrong answers
and your responses remain anonymous.}

\subsection*{Items}\label{items-9}
\addcontentsline{toc}{subsection}{Items}

\begin{longtable}[]{@{}
  >{\raggedright\arraybackslash}p{(\columnwidth - 2\tabcolsep) * \real{0.1667}}
  >{\raggedright\arraybackslash}p{(\columnwidth - 2\tabcolsep) * \real{0.8333}}@{}}
\toprule\noalign{}
\begin{minipage}[b]{\linewidth}\raggedright
Qlabel
\end{minipage} & \begin{minipage}[b]{\linewidth}\raggedright
question
\end{minipage} \\
\midrule\noalign{}
\endhead
\bottomrule\noalign{}
\endlastfoot
anth\_01 & I sometimes wonder if my computer deliberately runs more
slowly after I have shouted at it \\
anth\_02 & On occasions I feel that my computer/printer is being
deliberately awkward \\
anth\_03 & I sometimes wonder if my personal possessions appreciate it
when I have given them a good clean \\
anth\_04 & On occasion I feel that the weather conditions are being
deliberately bad in order to ruin a social event \\
anth\_05 & I do think that certain cars have a specific personality \\
anth\_06 & If I accidentally break one of my favourite possessions I
make sure that I apologise to it for my clumsiness \\
anth\_07 & I think that some trees are friendly while others have an air
of menace \\
anth\_08 & I sometimes think that if my computer/printer is made to feel
happy and/or wanted, then they will be less likely to malfunction \\
anth\_09 & I sometimes feel that the sea can be angry \\
anth\_10 & Part of the reason why I picked a new car/electrical item was
because when I first saw it I felt that it had a friendly personality \\
\end{longtable}

\subsection*{Coding}\label{coding-9}
\addcontentsline{toc}{subsection}{Coding}

This questionnaire follows our standard coding for agreement based
measures, with strongly disagree = 1, and strongly agree = 7.

\begin{longtable}[]{@{}
  >{\raggedright\arraybackslash}p{(\columnwidth - 12\tabcolsep) * \real{0.1429}}
  >{\raggedright\arraybackslash}p{(\columnwidth - 12\tabcolsep) * \real{0.1429}}
  >{\raggedright\arraybackslash}p{(\columnwidth - 12\tabcolsep) * \real{0.1429}}
  >{\raggedright\arraybackslash}p{(\columnwidth - 12\tabcolsep) * \real{0.1429}}
  >{\raggedright\arraybackslash}p{(\columnwidth - 12\tabcolsep) * \real{0.1429}}
  >{\raggedright\arraybackslash}p{(\columnwidth - 12\tabcolsep) * \real{0.1429}}
  >{\raggedright\arraybackslash}p{(\columnwidth - 12\tabcolsep) * \real{0.1429}}@{}}
\toprule\noalign{}
\endhead
\bottomrule\noalign{}
\endlastfoot
\textbf{1} & \textbf{2} & \textbf{3} & \textbf{4} & \textbf{5} &
\textbf{6} & \textbf{7} \\
strongly disagree & moderately disagree & slightly disagree & neither
agree nor disagree & slightly agree & moderately agree & strongly
agree \\
\end{longtable}

\texttt{No\ items\ are\ reverse\ coded}

\subsection{Scoring}\label{scoring-9}

The following variables were derived from this measure

\begin{tabu} to \linewidth {>{\raggedright\arraybackslash}p{3cm}>{\raggedright\arraybackslash}p{3cm}>{\raggedright\arraybackslash}p{3cm}>{\raggedright\arraybackslash}p{3cm}>{\raggedright\arraybackslash}p{3cm}>{\raggedright}X}
\toprule
Variable Name & Variable Label & Description & Variable Type & Source (Section) & Definition\\
\midrule
Anthropomorphism & anthropomorphism & Anthropomorphism score & numeric & Anthropomorphism & mean anth\_01: anth\_10\\
\bottomrule
\end{tabu}

\chapter{Mentalizing}\label{mentalizing}

\textbf{Cluster:} Cognitive Biases

\section{Measure}\label{measure-10}

The Mentalizing measure used was the EQ-Short (Wakabayashi et al.,
2006), which is an abbreviated version of the original Empathy Quotient
measure (Baron-Cohen \& Wheelwright, 2004).

\subsection{Modifications}\label{modifications-9}

In contrast with other agreement based measures we used, the selected
mentalizing measure used a unique scoring system, and as such was not
altered to be consistent with our other Likert measures.

The wording of the question was altered to reflect the manner of
administration (``select'' your answer instead of the original
``circle'')

\section{Implementation}\label{implementation-10}

\subsection*{Question wording}\label{question-wording-10}
\addcontentsline{toc}{subsection}{Question wording}

Participants read the following text:

\emph{Please read each of the following statements carefully and rate
how strongly you agree or disagree. There are no right or wrong answers
and your responses remain anonymous.}

\subsection*{Items}\label{items-10}
\addcontentsline{toc}{subsection}{Items}

\begin{longtable}[]{@{}
  >{\raggedright\arraybackslash}p{(\columnwidth - 2\tabcolsep) * \real{0.0708}}
  >{\raggedright\arraybackslash}p{(\columnwidth - 2\tabcolsep) * \real{0.9292}}@{}}
\toprule\noalign{}
\begin{minipage}[b]{\linewidth}\raggedright
Qlabel
\end{minipage} & \begin{minipage}[b]{\linewidth}\raggedright
question
\end{minipage} \\
\midrule\noalign{}
\endhead
\bottomrule\noalign{}
\endlastfoot
mnt\_01 & I can easily tell if someone else wants to enter a
conversation. \\
mnt\_02 & I really enjoy caring for other people \\
mnt\_03 & I find it hard to know what to do in a social situation \\
mnt\_04 & I often find it difficult to judge if something is rude or
polite \\
mnt\_05 & In a conversation, I tend to focus on my own thoughts rather
than on what my listener might be thinking \\
mnt\_06 & I can pick up quickly if someone says one thing but means
another. \\
mnt\_07 & It is hard for me to see why some things upset people so
much \\
mnt\_08 & I find it easy to put myself in somebody else's shoes \\
mnt\_09 & I am good at predicting how someone will feel. \\
mnt\_10 & I am quick to spot when someone in a group is feeling awkward
or uncomfortable \\
mnt\_11 & I can't always see why someone should have felt offended by a
remark \\
mnt\_12 & I don't tend to find social situations confusing. \\
mnt\_13 & Other people tell me I am good at understanding how they are
feeling and what they are thinking \\
mnt\_14 & I can easily tell if someone else is interested or bored with
what I am saying \\
mnt\_15 & Friends usually talk to me about their problems as they say
that I am very understanding. \\
mnt\_16 & I can sense if I am intruding, even if the other person
doesn't tell me \\
mnt\_17 & Other people often say that I am insensitive, though I don't
always see why \\
mnt\_18 & I can tune into how someone else feels rapidly and
intuitively. \\
mnt\_19 & I can easily work out what another person might want to talk
about \\
mnt\_20 & I can tell if someone is masking their true emotion \\
mnt\_21 & I am good at predicting what someone will do \\
mnt\_22 & I tend to get emotionally involved with a friend's problems \\
\end{longtable}

\subsection*{Coding}\label{coding-10}
\addcontentsline{toc}{subsection}{Coding}

The mentalizing measure follows a bespoke 4 point agreement scale and
uses a the following coding scheme:

\begin{longtable}[]{@{}
  >{\raggedright\arraybackslash}p{(\columnwidth - 6\tabcolsep) * \real{0.2838}}
  >{\raggedright\arraybackslash}p{(\columnwidth - 6\tabcolsep) * \real{0.2568}}
  >{\raggedright\arraybackslash}p{(\columnwidth - 6\tabcolsep) * \real{0.2162}}
  >{\raggedright\arraybackslash}p{(\columnwidth - 6\tabcolsep) * \real{0.2432}}@{}}
\toprule\noalign{}
\begin{minipage}[b]{\linewidth}\raggedright
definitely disagree
\end{minipage} & \begin{minipage}[b]{\linewidth}\raggedright
slightly disagree
\end{minipage} & \begin{minipage}[b]{\linewidth}\raggedright
slightly agree
\end{minipage} & \begin{minipage}[b]{\linewidth}\raggedright
definitely agree
\end{minipage} \\
\midrule\noalign{}
\endhead
\bottomrule\noalign{}
\endlastfoot
0 & 0 & 1 & 2 \\
\end{longtable}

\texttt{Items\ 03,\ 04,\ 05,\ 07,\ 11,\ and\ 17\ are\ reverse\ coded.}

\subsection{Scoring}\label{scoring-10}

The following variables are derived from this measure:

\begin{tabu} to \linewidth {>{\raggedright\arraybackslash}p{3cm}>{\raggedright\arraybackslash}p{3cm}>{\raggedright\arraybackslash}p{3cm}>{\raggedright\arraybackslash}p{3cm}>{\raggedright\arraybackslash}p{3cm}>{\raggedright}X}
\toprule
Variable Name & Variable Label & Description & Variable Type & Source (Section) & Definition\\
\midrule
Mentalizing & mentalizing & Mentalizing score & numeric & Mentalizing & mean mnt\_01: mnt\_22\\
\bottomrule
\end{tabu}

\chapter{Vividness of Mental Imagery}\label{vividness-of-mental-imagery}

\textbf{Cluster:} Cognitive Biases

\section{Measure}\label{measure-11}

The measure of vividness of mental imagery we use is the Vividness of
Mental Imagery scale Marks (1973).

\subsection*{Modifications}\label{modifications-10}
\addcontentsline{toc}{subsection}{Modifications}

\section{Implementation}\label{implementation-11}

For this questionnaire each subscale appeared on its own page with the
subscale specific text above

\subsection*{Question wording}\label{question-wording-11}
\addcontentsline{toc}{subsection}{Question wording}

Participants read the following text:

\begin{longtable}[]{@{}
  >{\raggedright\arraybackslash}p{(\columnwidth - 2\tabcolsep) * \real{0.0463}}
  >{\raggedright\arraybackslash}p{(\columnwidth - 2\tabcolsep) * \real{0.9537}}@{}}
\toprule\noalign{}
\begin{minipage}[b]{\linewidth}\raggedright
Items
\end{minipage} & \begin{minipage}[b]{\linewidth}\raggedright
Text
\end{minipage} \\
\midrule\noalign{}
\endhead
\bottomrule\noalign{}
\endlastfoot
& \\
vvi\_01 - vvi\_16 & For each scenario try to form a mental picture of
the people, objects, or setting. Consider carefully the vividness of
your visual imagery experience. Does some type of image come to mind?
Rate how vivid the image is using the 5-point scale. If you do not have
a visual image, rate vividness as `1'. Only use `5' for images that are
as lively and vivid as real seeing.

Please try to form the mental image of the following items with your
eyes \textbf{\emph{closed.}} \\
\end{longtable}

additional text was also presented prior to each subscale:

\begin{longtable}[]{@{}
  >{\raggedright\arraybackslash}p{(\columnwidth - 2\tabcolsep) * \real{0.2000}}
  >{\raggedright\arraybackslash}p{(\columnwidth - 2\tabcolsep) * \real{0.8000}}@{}}
\toprule\noalign{}
\begin{minipage}[b]{\linewidth}\raggedright
Subscale (items)
\end{minipage} & \begin{minipage}[b]{\linewidth}\raggedright
Text
\end{minipage} \\
\midrule\noalign{}
\endhead
\bottomrule\noalign{}
\endlastfoot
Relative or friend (vvi\_01 - vvi\_04) & For items 1 to 4, think of some
relative or friend whom you frequently see (but who is not with you at
present) and consider carefully the picture that comes before your
mind's eye. \\
Natural scene: Rising sun (vvi\_05 - vvi\_08) & Visualize a rising sun.
Consider carefully the picture that comes before your mind's eye. \\
Shop (vvi\_09 - vvi\_12) & Think of the front of a shop which you often
go to. Consider the picture that comes before your mind's eye. \\
Natural scene: Lake (vvi\_13 - vvi\_16) & Finally, think of a country
scene which involves trees, mountains, and a lake. Consider the picture
that comes before your mind's eye. \\
\end{longtable}

\subsection*{Items}\label{items-11}
\addcontentsline{toc}{subsection}{Items}

\begin{longtable}[]{@{}
  >{\raggedright\arraybackslash}p{(\columnwidth - 4\tabcolsep) * \real{0.0602}}
  >{\raggedright\arraybackslash}p{(\columnwidth - 4\tabcolsep) * \real{0.2030}}
  >{\raggedright\arraybackslash}p{(\columnwidth - 4\tabcolsep) * \real{0.7368}}@{}}
\toprule\noalign{}
\begin{minipage}[b]{\linewidth}\raggedright
Qlabel
\end{minipage} & \begin{minipage}[b]{\linewidth}\raggedright
Subscale
\end{minipage} & \begin{minipage}[b]{\linewidth}\raggedright
question
\end{minipage} \\
\midrule\noalign{}
\endhead
\bottomrule\noalign{}
\endlastfoot
vvi\_01 & Relative or friend & The exact contour of face, head,
shoulders, and body. \\
vvi\_02 & Relative or friend & Characteristic poses of head, attitudes
of body, etc. \\
vvi\_03 & Relative or friend & The precise carriage, length of step,
etc. in walking. \\
vvi\_04 & Relative or friend & The different colors worn in some
familiar clothes. \\
vvi\_05 & Natural scene: Rising sun & The sun is rising above the
horizon into a hazy sky. \\
vvi\_06 & Natural scene: Rising sun & The sky clears and surrounds the
sun with blueness. \\
vvi\_07 & Natural scene: Rising sun & Clouds. A storm blows up, with
flashes of lightening. \\
vvi\_08 & Natural scene: Rising sun & A rainbow appears. \\
vvi\_09 & Shop & The overall appearance of the shop from the opposite
side of the road. \\
vvi\_10 & Shop & A window display including colors, shape, and details
of individual items for sale. \\
vvi\_11 & Shop & You are near the entrance. The color, shape, and
details of the door. \\
vvi\_12 & Shop & You enter the shop and go to the counter. The counter
assistant serves you. Money changes hands. \\
vvi\_13 & Natural scene: Lake & The contours of the landscape. \\
vvi\_14 & Natural scene: Lake & The color and shape of the trees. \\
vvi\_15 & Natural scene: Lake & The color and shape of the lake. \\
vvi\_16 & Natural scene: Lake & A strong wind blows on the tree and on
the lake causing waves. \\
\end{longtable}

\subsection*{Coding}\label{coding-11}
\addcontentsline{toc}{subsection}{Coding}

This questionnaire follows a unique scoring scheme for vividness

\begin{longtable}[]{@{}
  >{\raggedright\arraybackslash}p{(\columnwidth - 8\tabcolsep) * \real{0.2513}}
  >{\raggedright\arraybackslash}p{(\columnwidth - 8\tabcolsep) * \real{0.1497}}
  >{\raggedright\arraybackslash}p{(\columnwidth - 8\tabcolsep) * \real{0.1497}}
  >{\raggedright\arraybackslash}p{(\columnwidth - 8\tabcolsep) * \real{0.0802}}
  >{\raggedright\arraybackslash}p{(\columnwidth - 8\tabcolsep) * \real{0.3690}}@{}}
\toprule\noalign{}
\begin{minipage}[b]{\linewidth}\raggedright
\textbf{1}
\end{minipage} & \begin{minipage}[b]{\linewidth}\raggedright
\textbf{2}
\end{minipage} & \begin{minipage}[b]{\linewidth}\raggedright
\textbf{3}
\end{minipage} & \begin{minipage}[b]{\linewidth}\raggedright
\textbf{4}
\end{minipage} & \begin{minipage}[b]{\linewidth}\raggedright
\textbf{5}
\end{minipage} \\
\midrule\noalign{}
\endhead
\bottomrule\noalign{}
\endlastfoot
Perfectly clear and as vivid as normal vision & Clear and reasonably
vivid & Moderately clear and vivid & Vague and dim & No image at all,
you only ``know'' that you are thinking of an object \\
\end{longtable}

\texttt{No\ items\ are\ reverse\ coded}

\subsection{Scoring}\label{scoring-11}

The following variables are derived from this measure:

\section{}\label{section-2}

\chapter{Dualism}\label{dualism}

\textbf{Cluster:} Cognitive Biases

\section{Measure}\label{measure-12}

The Dualism measure used is an abbreviation of The Dualism Scale by
Stanovich (1989). As with previous research the scale has been shortened
for brevity (Willard \& Cingl, 2017; Willard, Cingl, \& Norenzayan,
2020).

\subsection{Modifications.}\label{modifications.}

For consistency with the other agreement based measures used in the
project the anchors and response items were altered to be a 7-point
Likert scale

\section{Implementation}\label{implementation-12}

\subsection{Question wording}\label{question-wording-12}

Participants read the following text, adapted and shortened from
Stanovich (1989):

\emph{Please read each of the following statements carefully and rate
how strongly you agree or disagree. There are no right or wrong answers
and your responses remain anonymous}

\subsection*{Items}\label{items-12}
\addcontentsline{toc}{subsection}{Items}

\begin{longtable}[]{@{}
  >{\raggedright\arraybackslash}p{(\columnwidth - 2\tabcolsep) * \real{0.0548}}
  >{\raggedright\arraybackslash}p{(\columnwidth - 2\tabcolsep) * \real{0.9452}}@{}}
\toprule\noalign{}
\begin{minipage}[b]{\linewidth}\raggedright
Qlabel
\end{minipage} & \begin{minipage}[b]{\linewidth}\raggedright
question
\end{minipage} \\
\midrule\noalign{}
\endhead
\bottomrule\noalign{}
\endlastfoot
dua\_01 & The mind is not part of the brain but it affects the brain. \\
dua\_02 & When I imagine a scene in my mind, I am in a state that will
forever be beyond explanation by science. \\
dua\_03 & The mind is a special form of energy (currently unknown to
man) that is in contact with the brain and affects it. \\
dua\_04 & When people talk about their minds they are really just
talking about what their brains seem to be doing. \\
dua\_05 & The fact that I can know my own thought processes (that I can
introspect) means that my thought processes cannot be just brain
processes \\
dua\_06 & The mind is a nonmaterial substance that interact with the
brain to determine behavior \\
dua\_07 & Minds are inside brains but are not the same as brains. \\
dua\_08 & Some mental processes have no connection to brain processes \\
dua\_09 & Mental processes are the result of activity in the nervous
system \\
dua\_10 & The mind and the brain are two totally separate things \\
\end{longtable}

\subsection*{Coding}\label{coding-12}
\addcontentsline{toc}{subsection}{Coding}

This questionnaire follows our standard scoring for agreement based
measures, with strongly disagree = 1, and strongly agree = 7.

\begin{longtable}[]{@{}
  >{\raggedright\arraybackslash}p{(\columnwidth - 12\tabcolsep) * \real{0.1387}}
  >{\raggedright\arraybackslash}p{(\columnwidth - 12\tabcolsep) * \real{0.1533}}
  >{\raggedright\arraybackslash}p{(\columnwidth - 12\tabcolsep) * \real{0.1387}}
  >{\raggedright\arraybackslash}p{(\columnwidth - 12\tabcolsep) * \real{0.2044}}
  >{\raggedright\arraybackslash}p{(\columnwidth - 12\tabcolsep) * \real{0.1168}}
  >{\raggedright\arraybackslash}p{(\columnwidth - 12\tabcolsep) * \real{0.1314}}
  >{\raggedright\arraybackslash}p{(\columnwidth - 12\tabcolsep) * \real{0.1168}}@{}}
\toprule\noalign{}
\endhead
\bottomrule\noalign{}
\endlastfoot
\textbf{1} & \textbf{2} & \textbf{3} & \textbf{4} & \textbf{5} &
\textbf{6} & \textbf{7} \\
strongly disagree & moderately disagree & slightly disagree & neither
agree nor disagree & slightly agree & moderately agree & strongly
agree \\
\end{longtable}

\texttt{dua\_04\ and\ dua\_09\ are\ reverse\ coded}

\section{Scoring}\label{scoring-12}

\begin{tabu} to \linewidth {>{\raggedright\arraybackslash}p{3cm}>{\raggedright\arraybackslash}p{3cm}>{\raggedright\arraybackslash}p{3cm}>{\raggedright\arraybackslash}p{3cm}>{\raggedright\arraybackslash}p{3cm}>{\raggedright\arraybackslash}p{3cm}}
\toprule
Variable Name & Variable Label & Description & Variable Type & Source (Section) & Definition\\
\midrule
Dualism & dualism & Dualism score & numeric & Dualism & mean dua\_01: dua\_10\\
\bottomrule
\end{tabu}

\section{Script}\label{script}

\chapter{Promiscuous Teleology}\label{promiscuous-teleology}

\section{Measure}\label{measure-13}

The promiscuous teleology measure we use is the short form of the
Teleological Ideas about Nature Scale (TINS), which is an as yet
unpublished scale by Kelemen, Brown, Burnham \& Rottman (2023, in prep).

\texttt{PAGE\ WILL\ BE\ UPDATED\ WHEN\ THE\ SCALE\ IS\ PUBLISHED}

\subsection*{Modifications}\label{modifications-11}
\addcontentsline{toc}{subsection}{Modifications}

\section{Implementation}\label{implementation-13}

\subsection*{Question wording}\label{question-wording-13}
\addcontentsline{toc}{subsection}{Question wording}

Participants read the following text:

\emph{Please read each of the following statements carefully and state
to what extent you agree. There are no right or wrong answers and your
responses remain anonymous.}

\subsection*{Items}\label{items-13}
\addcontentsline{toc}{subsection}{Items}

\subsection*{Coding}\label{coding-13}
\addcontentsline{toc}{subsection}{Coding}

This questionnaire used the standard response scale for agreement

\texttt{No\ items\ are\ reverse\ coded}

\subsection{Scoring}\label{scoring-13}

The following variables are derived from this measure:

\begin{tabu} to \linewidth {>{\raggedright\arraybackslash}p{3cm}>{\raggedright\arraybackslash}p{3cm}>{\raggedright\arraybackslash}p{3cm}>{\raggedright\arraybackslash}p{3cm}>{\raggedright\arraybackslash}p{3cm}>{\raggedright}X}
\toprule
Variable Name & Variable Label & Description & Variable Type & Source (Section) & Definition\\
\midrule
Promiscuous Teleology & promisc\_teleology & Promiscuous Teleology Score & numeric & Promiscuous Teleology & mean tel\_01: tel\_14\\
\bottomrule
\end{tabu}

\section{Script}\label{script-1}

\chapter{Thinking Style}\label{thinking-style}

\textbf{Cluster:} Cognitive Style

\section{Measure}\label{measure-14}

We use the original Cognitive Reflection Task by Frederick (2005), and
an updated version, the Cognitive Reflection Task 2,from Thomson \&
Oppenheimer (2016).

\subsection*{Modifications}\label{modifications-12}
\addcontentsline{toc}{subsection}{Modifications}

\section{Implementation}\label{implementation-14}

\subsection*{Question wording}\label{question-wording-14}
\addcontentsline{toc}{subsection}{Question wording}

Participants read the following text:

\emph{In the following section you will be asked a series of questions.
Please do your best to answer as accurately as possible.}

\subsection*{Items}\label{items-14}
\addcontentsline{toc}{subsection}{Items}

\begin{longtable}[]{@{}
  >{\raggedright\arraybackslash}p{(\columnwidth - 2\tabcolsep) * \real{0.2917}}
  >{\raggedright\arraybackslash}p{(\columnwidth - 2\tabcolsep) * \real{0.7083}}@{}}
\toprule\noalign{}
\begin{minipage}[b]{\linewidth}\raggedright
Qlabel
\end{minipage} & \begin{minipage}[b]{\linewidth}\raggedright
question
\end{minipage} \\
\midrule\noalign{}
\endhead
\bottomrule\noalign{}
\endlastfoot
crt\_01 & You're running a race and you pass the person in second place,
what place are you in? \\
crt\_02 & A farmer had 15 sheep and all but 8 died. How many are
left? \\
crt\_03 & Emily's father has three daughters. The first two are named
April and May. What is the third daughter's name? \\
crt\_04 & How many cubic feet of dirt are there in a hole that is 3'
deep x 3' wide x 3' long? \\
crt\_05 & A bat and a ball cost \$1.10 in total. The bat costs \$1.00
more than the ball. How much does the ball cost? \\
crt\_06 & If it takes 5 machines 5 min to make 5 widgets, how long would
it take 100 machines to make 100 widgets? \\
crt\_07 & In a lake, there is a patch of lily pads. Every day, the patch
doubles in size. If it takes 48 days for the patch to cover the entire
lake, how long would it take for the patch to cover half of the lake? \\
\end{longtable}

\subsection*{Coding}\label{coding-14}
\addcontentsline{toc}{subsection}{Coding}

This questionnaire uses open responses for \texttt{crt} items.

For scoring responses are labelled as either ``Correct'', ``Intuitive''
or ``Incorrect''. Correct and intuitive answers are as follows:

`we also code any typos/variations accordingly for all items, see rows
1:3 as illustration)

\begin{longtable}[]{@{}
  >{\raggedright\arraybackslash}p{(\columnwidth - 4\tabcolsep) * \real{0.2055}}
  >{\raggedright\arraybackslash}p{(\columnwidth - 4\tabcolsep) * \real{0.4247}}
  >{\raggedright\arraybackslash}p{(\columnwidth - 4\tabcolsep) * \real{0.3699}}@{}}
\toprule\noalign{}
\begin{minipage}[b]{\linewidth}\raggedright
Item
\end{minipage} & \begin{minipage}[b]{\linewidth}\raggedright
Correct (including variations/typos)
\end{minipage} & \begin{minipage}[b]{\linewidth}\raggedright
Intuitive (!!CHECK THESE!!)
\end{minipage} \\
\midrule\noalign{}
\endhead
\bottomrule\noalign{}
\endlastfoot
crt\_01 & second (or 2nd, 2 or any variation therein) & first (or 1st, 1
or any variation of) \\
crt\_02 & 8 (or eight or other variations of/typos) & 7 (or seven or
variations of) \\
crt\_03 & Emily (or typos of) & June (or typos) \\
crt\_04 & 0 & 3 \\
crt\_04 & 5 cents & 10 cents \\
crt\_05 & 5 minutes & 1 minute \\
crt\_06 & 47 days & 24 days \\
\end{longtable}

\subsection{Scoring}\label{scoring-14}

The following variables are derived from this measure:

\begin{tabu} to \linewidth {>{\raggedright\arraybackslash}p{3cm}>{\raggedright\arraybackslash}p{3cm}>{\raggedright\arraybackslash}p{3cm}>{\raggedright\arraybackslash}p{3cm}>{\raggedright\arraybackslash}p{3cm}>{\raggedright\arraybackslash}p{3cm}}
\toprule
Variable Name & Variable Label & Description & Variable Type & Source (Section) & Definition\\
\midrule
Cognitive Style (correct) & crt\_correct & Proportion correct cogntive styles questions & numeric/binomial & Thinking Style & The mean (proportion) of crt\_01: crt\_07 that are correct\\
Cognitive Style (intuitive) & crt\_type & Proportion intuitive cogntive styles questions & numeric/binomial & Thinking Style & The mean (proportion) of crt\_01:crt\_07 that are intuitive\\
\bottomrule
\end{tabu}

\chapter{Existential Insecurity}\label{existential-insecurity-1}

\textbf{Cluster:} Motivational

\section{Measure}\label{measure-15}

The initial existential insecurity measures were derived from those used
by Baimel et al. (2022) and by Willard \& Cingl (2017).

\subsection*{Modifications}\label{modifications-13}
\addcontentsline{toc}{subsection}{Modifications}

The food security measure from Baimel et al. (2022) was combined with
financial and physical security from Willard \& Cingl (2017), these were
asked for `recently' and `for the forthcoming months'.

In this wave we only included questions pertaining to childhood.

\section{Implementation}\label{implementation-15}

\subsection*{Question wording}\label{question-wording-15}
\addcontentsline{toc}{subsection}{Question wording}

The question wording was as follows:

\begin{longtable}[]{@{}
  >{\raggedright\arraybackslash}p{(\columnwidth - 2\tabcolsep) * \real{0.2361}}
  >{\raggedright\arraybackslash}p{(\columnwidth - 2\tabcolsep) * \real{0.7639}}@{}}
\toprule\noalign{}
\begin{minipage}[b]{\linewidth}\raggedright
Questions
\end{minipage} & \begin{minipage}[b]{\linewidth}\raggedright
Question Text
\end{minipage} \\
\midrule\noalign{}
\endhead
\bottomrule\noalign{}
\endlastfoot
es\_15 - es\_21 & \emph{When growing up did you:} \\
\end{longtable}

\subsection*{Items}\label{items-15}
\addcontentsline{toc}{subsection}{Items}

The question items were as follows:

\textbf{Existential Security}

\begin{longtable}[]{@{}
  >{\raggedright\arraybackslash}p{(\columnwidth - 2\tabcolsep) * \real{0.1806}}
  >{\raggedright\arraybackslash}p{(\columnwidth - 2\tabcolsep) * \real{0.8194}}@{}}
\toprule\noalign{}
\begin{minipage}[b]{\linewidth}\raggedright
Qlabel
\end{minipage} & \begin{minipage}[b]{\linewidth}\raggedright
question
\end{minipage} \\
\midrule\noalign{}
\endhead
\bottomrule\noalign{}
\endlastfoot
es\_15 & worry that your household may not able to buy or produce enough
food to eat? \\
es\_16 & feel that your household may not be able to afford to buy items
you need \\
es\_17 & worry that your household may not have enough money \\
es\_18 & worry about members of your household losing their job \\
es\_19 & feel you will be safe walking alone in your local area after
dark \\
es\_20 & worry about being burgled \\
es\_21 & worry about being a victim of violent crime \\
\end{longtable}

\subsection*{Coding}\label{coding-15}
\addcontentsline{toc}{subsection}{Coding}

Items on the existential security subscale use the following frequency
based Likert scale:

\begin{longtable}[]{@{}
  >{\raggedright\arraybackslash}p{(\columnwidth - 12\tabcolsep) * \real{0.1429}}
  >{\raggedright\arraybackslash}p{(\columnwidth - 12\tabcolsep) * \real{0.1429}}
  >{\raggedright\arraybackslash}p{(\columnwidth - 12\tabcolsep) * \real{0.1429}}
  >{\raggedright\arraybackslash}p{(\columnwidth - 12\tabcolsep) * \real{0.1429}}
  >{\raggedright\arraybackslash}p{(\columnwidth - 12\tabcolsep) * \real{0.1429}}
  >{\raggedright\arraybackslash}p{(\columnwidth - 12\tabcolsep) * \real{0.1429}}
  >{\raggedright\arraybackslash}p{(\columnwidth - 12\tabcolsep) * \real{0.1429}}@{}}
\toprule\noalign{}
\endhead
\bottomrule\noalign{}
\endlastfoot
\textbf{1} & \textbf{2} & \textbf{3} & \textbf{4} & \textbf{5} &
\textbf{6} & \textbf{7} \\
never & very infrequently & infrequently & sometimes & frequently & very
frequently & always \\
\end{longtable}

\texttt{es\_19\ is\ reverse\ coded}

\subsection{Scoring}\label{scoring-15}

The following variables are derived form this measure:

\begin{tabu} to \linewidth {>{\raggedright\arraybackslash}p{3cm}>{\raggedright\arraybackslash}p{3cm}>{\raggedright\arraybackslash}p{3cm}>{\raggedright\arraybackslash}p{3cm}>{\raggedright\arraybackslash}p{3cm}>{\raggedright\arraybackslash}p{3cm}}
\toprule
Variable Name & Variable Label & Description & Variable Type & Source (Section) & Definition\\
\midrule
Existential Security (upbringing) & ex\_sec\_upb & Upbringing existential security score & numeric & Existential Security & mean es\_15: es\_21\\
\bottomrule
\end{tabu}

\chapter{Tolerance of Ambiguity}\label{tolerance-of-ambiguity}

\textbf{Cluster:} Cognitive Styles

\section{Measure}\label{measure-16}

The measure used for tolerance of ambiguity is the Multiple Stimulus
Types Ambiguity Tolerance Scale- II (MSTAT-II) by McLain (2009) .

\subsection*{Modifications}\label{modifications-14}
\addcontentsline{toc}{subsection}{Modifications}

\section{Implementation}\label{implementation-16}

\subsection*{Question wording}\label{question-wording-16}
\addcontentsline{toc}{subsection}{Question wording}

Participants read the following text:

\emph{Please read each of the following statements carefully and state
to what extent you agree with each. There are no right or wrong answers
and your responses remain anonymous.}

\subsection*{Items}\label{items-16}
\addcontentsline{toc}{subsection}{Items}

\begin{longtable}[]{@{}
  >{\raggedright\arraybackslash}p{(\columnwidth - 2\tabcolsep) * \real{0.1806}}
  >{\raggedright\arraybackslash}p{(\columnwidth - 2\tabcolsep) * \real{0.8194}}@{}}
\toprule\noalign{}
\begin{minipage}[b]{\linewidth}\raggedright
Qlabel
\end{minipage} & \begin{minipage}[b]{\linewidth}\raggedright
question
\end{minipage} \\
\midrule\noalign{}
\endhead
\bottomrule\noalign{}
\endlastfoot
at\_01 & I don't tolerate ambiguous situations well \\
at\_02 & I would rather avoid solving a problem that must be viewed from
several different perspectives \\
at\_03 & I try to avoid situations that are ambiguous \\
at\_04 & I prefer familiar situations to new ones \\
at\_05 & Problems that cannot be considered from just one point of view
are a little threatening \\
at\_06 & I avoid situations that are too complicated for me to easily
understand \\
at\_07 & I am tolerant of ambiguous situations \\
at\_08 & I enjoy tackling problems that are complex enough to be
ambiguous \\
at\_09 & I try to avoid problems that don't seem to have only one
``best'' solution \\
at\_10 & I generally prefer novelty over familiarity \\
at\_11 & I dislike ambiguous situations \\
at\_12 & I find it hard to make a choice when the outcome is
uncertain \\
at\_13 & I prefer a situation in which there is some ambiguity \\
\end{longtable}

\subsection*{Coding}\label{coding-16}
\addcontentsline{toc}{subsection}{Coding}

This questionnaire uses our standard response scale for agreement,

\begin{longtable}[]{@{}
  >{\raggedright\arraybackslash}p{(\columnwidth - 12\tabcolsep) * \real{0.1429}}
  >{\raggedright\arraybackslash}p{(\columnwidth - 12\tabcolsep) * \real{0.1429}}
  >{\raggedright\arraybackslash}p{(\columnwidth - 12\tabcolsep) * \real{0.1429}}
  >{\raggedright\arraybackslash}p{(\columnwidth - 12\tabcolsep) * \real{0.1429}}
  >{\raggedright\arraybackslash}p{(\columnwidth - 12\tabcolsep) * \real{0.1429}}
  >{\raggedright\arraybackslash}p{(\columnwidth - 12\tabcolsep) * \real{0.1429}}
  >{\raggedright\arraybackslash}p{(\columnwidth - 12\tabcolsep) * \real{0.1429}}@{}}
\toprule\noalign{}
\begin{minipage}[b]{\linewidth}\raggedright
\textbf{1}
\end{minipage} & \begin{minipage}[b]{\linewidth}\raggedright
\textbf{2}
\end{minipage} & \begin{minipage}[b]{\linewidth}\raggedright
\textbf{3}
\end{minipage} & \begin{minipage}[b]{\linewidth}\raggedright
\textbf{4}
\end{minipage} & \begin{minipage}[b]{\linewidth}\raggedright
\textbf{5}
\end{minipage} & \begin{minipage}[b]{\linewidth}\raggedright
\textbf{6}
\end{minipage} & \begin{minipage}[b]{\linewidth}\raggedright
\textbf{7}
\end{minipage} \\
\midrule\noalign{}
\endhead
\bottomrule\noalign{}
\endlastfoot
strongly disagree & moderately disagree & slightly disagree & neither
agree nor disagree & slightly agree & moderately agree & strongly
agree \\
\end{longtable}

\texttt{at\_01,\ at\_02,\ at\_03,\ at\_04,\ at\_05,\ at\_06,\ at\_09,\ at\_11,\ and\ at\_12\ are\ reverse\ coded}

\subsection{Scoring}\label{scoring-16}

The following variables are derived from this measure:

\begin{tabu} to \linewidth {>{\raggedright\arraybackslash}p{3cm}>{\raggedright\arraybackslash}p{3cm}>{\raggedright\arraybackslash}p{3cm}>{\raggedright\arraybackslash}p{3cm}>{\raggedright\arraybackslash}p{3cm}>{\raggedright\arraybackslash}p{3cm}}
\toprule
Variable Name & Variable Label & Description & Variable Type & Source (Section) & Definition\\
\midrule
Tolerance of Ambiguity & ambiguity\_tol & Tolerance of ambiguity  score & numeric & Tolerance of Ambiguity & The mean of at\_01: at\_13\\
\bottomrule
\end{tabu}

\chapter{Importance of Rationality}\label{importance-of-rationality}

\textbf{Cluster:} Cognitive Style

\section{Measure}\label{measure-17}

Here we include the Importance of Rationality Scale from Ståhl, Zaal, \&
Skitka (2016)

\subsection*{Modifications}\label{modifications-15}
\addcontentsline{toc}{subsection}{Modifications}

\section{Implementation}\label{implementation-17}

\subsection*{Question wording}\label{question-wording-17}
\addcontentsline{toc}{subsection}{Question wording}

Participants read the following text:

\emph{Please read each of the following statements carefully and state
to what extent you agree. There are no right or wrong answers and your
responses remain anonymous.}

\textbf{INSERT}

\subsection*{Items}\label{items-17}
\addcontentsline{toc}{subsection}{Items}

\begin{longtable}[]{@{}
  >{\raggedright\arraybackslash}p{(\columnwidth - 2\tabcolsep) * \real{0.3194}}
  >{\raggedright\arraybackslash}p{(\columnwidth - 2\tabcolsep) * \real{0.6806}}@{}}
\toprule\noalign{}
\endhead
\bottomrule\noalign{}
\endlastfoot
imp\_01 & It is important to me personally to be skeptical about claims
that are not backed up by evidence. \\
imp\_02 & It is important to me personally to remain rational and
levelheaded even in heated arguments. \\
imp\_03 & It is important to me personally to examine traditionally held
beliefs using logic and evidence. \\
imp\_04 & It is important to me personally that I can justify my beliefs
using rational arguments and evidence. \\
imp\_05 & It is important to me personally to critically examine my
long-held beliefs. \\
imp\_06 & It is important to me personally to be a rational person. \\
\end{longtable}

\subsection*{Coding}\label{coding-17}
\addcontentsline{toc}{subsection}{Coding}

This questionnaire uses our standard agreement scale.

\begin{longtable}[]{@{}
  >{\raggedright\arraybackslash}p{(\columnwidth - 12\tabcolsep) * \real{0.1429}}
  >{\raggedright\arraybackslash}p{(\columnwidth - 12\tabcolsep) * \real{0.1429}}
  >{\raggedright\arraybackslash}p{(\columnwidth - 12\tabcolsep) * \real{0.1429}}
  >{\raggedright\arraybackslash}p{(\columnwidth - 12\tabcolsep) * \real{0.1429}}
  >{\raggedright\arraybackslash}p{(\columnwidth - 12\tabcolsep) * \real{0.1429}}
  >{\raggedright\arraybackslash}p{(\columnwidth - 12\tabcolsep) * \real{0.1429}}
  >{\raggedright\arraybackslash}p{(\columnwidth - 12\tabcolsep) * \real{0.1429}}@{}}
\toprule\noalign{}
\begin{minipage}[b]{\linewidth}\raggedright
\textbf{1}
\end{minipage} & \begin{minipage}[b]{\linewidth}\raggedright
\textbf{2}
\end{minipage} & \begin{minipage}[b]{\linewidth}\raggedright
\textbf{3}
\end{minipage} & \begin{minipage}[b]{\linewidth}\raggedright
\textbf{4}
\end{minipage} & \begin{minipage}[b]{\linewidth}\raggedright
\textbf{5}
\end{minipage} & \begin{minipage}[b]{\linewidth}\raggedright
\textbf{6}
\end{minipage} & \begin{minipage}[b]{\linewidth}\raggedright
\textbf{7}
\end{minipage} \\
\midrule\noalign{}
\endhead
\bottomrule\noalign{}
\endlastfoot
strongly disagree & moderately disagree & slightly disagree & neither
agree nor disagree & slightly agree & moderately agree & strongly
agree \\
\end{longtable}

\texttt{No\ items\ are\ reverse\ coded}

\subsection*{Scoring}\label{scoring-17}
\addcontentsline{toc}{subsection}{Scoring}

The following variables are derived form this measure:

\begin{tabu} to \linewidth {>{\raggedright\arraybackslash}p{3cm}>{\raggedright\arraybackslash}p{3cm}>{\raggedright\arraybackslash}p{3cm}>{\raggedright\arraybackslash}p{3cm}>{\raggedright\arraybackslash}p{3cm}>{\raggedright\arraybackslash}p{3cm}}
\toprule
Variable Name & Variable Label & Description & Variable Type & Source (Section) & Definition\\
\midrule
Importance of Rationality & rationality\_imp & Importance of rationality score & numeric & Importance of Rationality & The mean of imp\_01 : imp\_06\\
\bottomrule
\end{tabu}

\part{Belief Measures}

The variables defined here are derived from the following measures for
which detailed information can be found on their respective pages:

\begin{itemize}
\tightlist
\item
  \href{issp.qmd}{ISSP}
\item
  \href{religiousidentity.qmd}{Religious Identity}
\item
  \href{religiouspractice.qmd}{Religious Practice}
\item
  \href{supernaturalbelief.qmd}{Supernatural Belief}
\item
  \href{generalbelief.qmd}{General Belief}
\item
  \href{atheistmembership.qmd}{Atheist Membership}
\end{itemize}

\section*{Belief Variables}\label{belief-variables}
\addcontentsline{toc}{section}{Belief Variables}

\markright{Belief Variables}

\subsection*{Belief in God}\label{belief-in-god}
\addcontentsline{toc}{subsection}{Belief in God}

\begin{tabu} to \linewidth {>{\raggedright\arraybackslash}p{3cm}>{\raggedright\arraybackslash}p{3cm}>{\raggedright\arraybackslash}p{3cm}>{\raggedright\arraybackslash}p{3cm}>{\raggedright\arraybackslash}p{3cm}>{\raggedright\arraybackslash}p{3cm}}
\toprule
Variable Name & Variable Label & Description & Variable Type & Definition & Response Options\\
\midrule
Belief in God (categorical) & belief\_god\_cat & Categorical & Categorical Response to belief in god options & issp\_01 default & 1. I don't believe in God

2. I don't know whether there is a God, and I don't believe there is any way to find out

3. I don't believe in a personal God, but I do believe in a Higher Power of some kind

4. I find myself believing in God some of the time, but not at others

5. While I have doubts, I feel that I do believe in God.

6. I know God really exists and I have no doubt about it.

7. Don't know\\
Non-Belief Identity & belief\_ident & Categorical & Cattegorical response to non-believers identity options & issp\_02 default & 1. Spiritual but not religious
2.  Seeker
3.  Non-religious
4.  Atheist
5.  Agnostic
6.  Humanist
7.  Sceptic
8.  Free thinker
9.  Rationalist
10. Secular
11. Christian
12. Muslim
13. Jewish
14. Buddhist
15. Hindu
16. Daoist
17. Shinto
18. Confucian
19. Other (please specify)\\
Belief in God (binary) & belief\_god\_bin & binary (dummy) & Categorisation of "believers" and "non-believers" from issp\_01 & dummy variable for yes responses to either 5 \& 6 in issp\_01 & 0,1\\
Belief in God (ordinal) & belief\_god\_con & continuous/ ordinal & Likert measure of agreement with belief in god statement. & belief\_01 & 1 to 7\\
Agnosticism (ordinal) & agnosticism\_cont & continuous/ ordinal & Likert scale measure of agreement with agnosticism statement. & agn\_01 & 1 to 7\\
\addlinespace
Agnostic Identity (binary) & agnosticism\_bin & binary & Categorisation of "agnostics" from issp\_02 & dummy/index variable for response 5 in issp\_02 & 0,1\\
\bottomrule
\end{tabu}

\subsection*{Religiosity}\label{religiosity}
\addcontentsline{toc}{subsection}{Religiosity}

\begin{tabu} to \linewidth {>{\raggedright\arraybackslash}p{3cm}>{\raggedright\arraybackslash}p{3cm}>{\raggedright\arraybackslash}p{3cm}>{\raggedright\arraybackslash}p{3cm}>{\raggedright\arraybackslash}p{3cm}>{\raggedright\arraybackslash}p{3cm}}
\toprule
Variable Name & Variable Label & Description & Variable Type & Definition & Response Options\\
\midrule
Religious Identification & relig\_id\_1 & binary & Yes/no belong to a religion & rid\_01 default & 0,1\\
Religious Identity & relig\_id\_2 & categorical & Which religion category & rid\_02 default & 1. Protestant
2. Catholic
3. Orthodox (e.g. Greek Orthodox, Russian Orthodox)
4. Sunni
5. Shiite
6.Buddhist
7. Confucian
8. Daoist
9. Hindu
10. Jewish
11. Shinto
12. Other (please \vphantom{2} specify)\\
Religious Attendance & relig\_attend & continuous/ ordinal & Frequency of religious attendence likert scale & rp\_01 default & 1. More than once a week
2. Once a week
3. Once a month
4. Only on special holy days
5. Once a year
6. Less often
7. Never, practically never\\
Prayer Frequency & prayer\_freq & continuous/ ordinal & Frequency of prayer likert scale & rp\_02 default & 1. Several times a day
2. Once a day
3. Several times each week
4. Only when attending religious services
5. Only on special holy days
6. Once a year
7. Less often
8. Never, practically never\\
Religious Objects & relig\_object & categorical & Presence of religious objects & rp\_03 default & 1. Yes, for religious reasons
2. Yes, for non-religious reasons
3. No\\
\addlinespace
Anti-religiosity & anti\_relig & continuous/ ordinal & Agreement with anti-religiosity statement & ar\_01 default & 1 to 7\\
Primary Caregiver Religiosity & caregiver\_relig & binary & Yes/no belong to a religion & rid\_03 default & 0,1\\
Primary Caregiver Religios Identity & caregiver\_relig\_id & categorical & Which religion category & rid\_04 default & 1. Protestant
2. Catholic
3. Orthodox (e.g. Greek Orthodox, Russian Orthodox)
4. Sunni
5. Shiite
6.Buddhist
7. Confucian
8. Daoist
9. Hindu
10. Jewish
11. Shinto
12. Other (please \vphantom{1} specify)\\
Additional Caregiver Religiousity & caregiver2\_relig & binary & Yes/no belong to a religion & rid\_05 default & 0,1\\
Additional Caregiver Religiosity  Identity & caregiver2\_relig\_id & categorical & Which religion category & rid\_06 default & 1. Protestant
2. Catholic
3. Orthodox (e.g. Greek Orthodox, Russian Orthodox)
4. Sunni
5. Shiite
6.Buddhist
7. Confucian
8. Daoist
9. Hindu
10. Jewish
11. Shinto
12. Other (please specify)\\
\bottomrule
\end{tabu}

\subsection*{Atheism}\label{atheism}
\addcontentsline{toc}{subsection}{Atheism}

\begin{tabu} to \linewidth {>{\raggedright\arraybackslash}p{3cm}>{\raggedright\arraybackslash}p{3cm}>{\raggedright\arraybackslash}p{3cm}>{\raggedright\arraybackslash}p{3cm}>{\raggedright\arraybackslash}p{3cm}>{\raggedright\arraybackslash}p{3cm}}
\toprule
Variable Name & Variable Label & Description & Variable Type & Definition & Response Options\\
\midrule
Atheist Membership & atheist\_mem & Yes/no to membership of an atheist org. & binary & am\_01 response & 0,1\\
Atheist Identity & atheist\_id & Categorisation of "athetists" from issp\_02 & binary & dummy variable from option 4 for issp\_02 & 0,1\\
\bottomrule
\end{tabu}

\subsection*{Meta-Belief}\label{meta-belief}
\addcontentsline{toc}{subsection}{Meta-Belief}

\begin{tabu} to \linewidth {>{\raggedright\arraybackslash}p{3cm}>{\raggedright\arraybackslash}p{3cm}>{\raggedright\arraybackslash}p{3cm}>{\raggedright\arraybackslash}p{3cm}>{\raggedright\arraybackslash}p{3cm}>{\raggedright\arraybackslash}p{3cm}}
\toprule
Variable Name & Variable Label & Description & Variable Type & Definition & Response Options\\
\midrule
Possibility of Knowing (God) & god\_knowing & continuous/ ordinal & Agreement with: "It is not possible to know if God exists" & belief\_02 default & 1 to 7\\
Confidence in Belief & belief\_confidence & continuous/ ordinal & Agreement with: "I am confident that my beliefs about God's existence are the right ones" & conf\_01 default & 1 to 7\\
Apatheism (God) & apatheism\_god & continuous/ ordinal & Agreement with:  "Whether or not God exists is a question that doesn't interest me much" & apth\_01 default & 1 to 7\\
Apatheism (life purpose) & apatheism\_purpose & continuous/ ordinal & Agreement with: "Whether or not there is an ultimate purpose to life is a question that doesn't interest me much." & apth\_02 default & 1 to 7\\
Apatheism (combined) & apatheism & continuous/ & Mean of apatheism\_god + apatheism\_purpose & mean of apth\_01 and apth\_02 & 1 to 7\\
\addlinespace
Possibility of Truth & truth\_possibility & continuous/ ordinal & Agreement with: "For most things in the world, we will never be able to discover the real objective truth." & mean\_01 default & 1 to 7\\
Naturalism & naturalism & binary & Lack of agreement for any supernatural belief items & dummy variable = 0 for cases where any of snb items (except 16, 17) are > 4 (agreement or neither) & 1 to 7\\
\bottomrule
\end{tabu}

\subsection*{Supernatural Belief}\label{supernatural-belief}
\addcontentsline{toc}{subsection}{Supernatural Belief}

Supernatural Belief measures use our 1-7 agreement scale:

\begin{tabu} to \linewidth {>{\centering\arraybackslash}p{2.5cm}>{\centering\arraybackslash}p{2.5cm}>{\centering\arraybackslash}p{2.5cm}>{\centering\arraybackslash}p{2.5cm}>{\centering\arraybackslash}p{2.5cm}>{\centering\arraybackslash}p{2.5cm}>{\centering\arraybackslash}p{2.5cm}}
\toprule
Strongly Disagree & Moderately Disagree & Slightly Disagree & Neither Agree nor Disagree & Slightly Agree & Moderately Agree & Strongly Agree\\
\midrule
1 & 2 & 3 & 4 & 5 & 6 & 7\\
\bottomrule
\end{tabu}

\subsubsection*{Supernatural Belief
Variables}\label{supernatural-belief-variables}
\addcontentsline{toc}{subsubsection}{Supernatural Belief Variables}

\begin{tabu} to \linewidth {>{\raggedright\arraybackslash}p{3cm}>{\raggedright\arraybackslash}p{3cm}>{\raggedright\arraybackslash}p{3cm}>{\raggedright\arraybackslash}p{3cm}>{\raggedright\arraybackslash}p{3cm}>{\raggedright\arraybackslash}p{3cm}}
\toprule
Variable Name & Variable Label & Description & Variable Type & Definition & Response Options\\
\midrule
Afterlife Existence & afterlife\_exist & Agreement with: "There is some sort of life after death" & continuous/ ordinal & snb\_01 default & 1 to 7\\
Afterlife Punishment & afterlife\_punish & Agreement with: "Some people will be punished after they die." & continuous/ ordinal & snb\_13 default & 1 to 7\\
Afterlife Reward & afterlife\_reward & Agreement with: "Some people will be rewarded after they die" & continuous/ ordinal & snb\_14 default & 1 to 7\\
Reincarnation & reincarnation & Agreement with: "Sometime after I die, I expect that I'll be born again in another body." & continuous/ ordinal & snb\_02 default & 1 to 7\\
Astrology & astrology & Agreement with: "The positions of the stars and planets affect people's lives" & continuous/ ordinal & snb\_03 default & 1 to 7\\
\addlinespace
Mystical People & msystical\_people & Agreement with: "Some people have mystical powers (e.g. to heal, harm, or bring good luck)" & continuous/ ordinal & snb\_04 default & 1 to 7\\
Mystical Objects & mystical\_objects & Agreement with: "Some objects have mystical powers (e.g. to heal, harm, or bring good luck)" & continuous/ ordinal & snb\_05 default & 1 to 7\\
Good and Evil & good\_evil & Agreement with: "There are underlying forces of good and evil in this world." & continuous/ ordinal & snb\_06 default & 1 to 7\\
Universal spirit or life force & lifeforce & Agreement with: "There exists a universal spirit or life force." & continuous/ ordinal & snb\_07 default & 1 to 7\\
Karma & karma & Agreement with: "There is a power in the universe that causes good things to happen to people who behave morally and bad things to happen to people who behave immorally." & continuous/ ordinal & snb\_08 default & 1 to 7\\
\addlinespace
Fate & fate & Agreement with: "Most significant life events are meant to be and happen for a reason." & continuous/ ordinal & snb\_09 default & 1 to 7\\
Supernatural Being Existence & supernat\_beings & Agreement with: "Supernatural beings of some kind exist" & continuous/ ordinal & snb\_10 default & 1 to 7\\
Good Supernatural Beings & good\_beings & Agreement with: "There exist supernatural beings that are good/kind (e.g. COUNTRY SPECIFIC)" & continuous/ ordinal & snb\_11 default & 1 to 7\\
Harmful Supernatural Beings & harmful\_beings & Agreement with: "There exist supernatural beings that are harmful. (e.g COUNTRY SPECIFIC)" & continuous/ ordinal & snb\_12 default & 1 to 7\\
Evil Eye & evil\_eye & Agreement with: "Making other people envious of you can cause illness or misfortune." & continuous/ ordinal & snb\_15 default & 1 to 7\\
\addlinespace
Spiritual Experience (personal) & spiritual\_force & Category response to: "Have you ever felt as though you were connected to a powerful spiritual force?" & continuous/ ordinal & snb\_16 default & 1. Yes, I've had an experience like this.
2. I've had an experience like this, but I didn't associate it with a spiritual force.
3. No, I've never had an experience like this.\\
Lucky Object & lucky\_objects & Yes/No to: "Do you carry any objects for luck or protection?" & binary & snb\_17 default & 0, 1\\
\bottomrule
\end{tabu}

\chapter{ISSP}\label{issp}

\textbf{Cluster:} Measuring Belief

\section{Measure}\label{measure-18}

\subsection*{Modifications}\label{modifications-16}
\addcontentsline{toc}{subsection}{Modifications}

Note ISSP\_03 was not originally an ISSP question. This is a question
created by us and added to this section for practical purposes.

\section{Implementation}\label{implementation-18}

\texttt{issp\_02\ only\ appears\ if\ issp\_01\ =\ 1\ or\ 2}

\begin{longtable}[]{@{}
  >{\raggedright\arraybackslash}p{(\columnwidth - 4\tabcolsep) * \real{0.0405}}
  >{\raggedright\arraybackslash}p{(\columnwidth - 4\tabcolsep) * \real{0.6453}}
  >{\raggedright\arraybackslash}p{(\columnwidth - 4\tabcolsep) * \real{0.3108}}@{}}
\toprule\noalign{}
\begin{minipage}[b]{\linewidth}\raggedright
Qlabel
\end{minipage} & \begin{minipage}[b]{\linewidth}\raggedright
Question
\end{minipage} & \begin{minipage}[b]{\linewidth}\raggedright
Response options
\end{minipage} \\
\midrule\noalign{}
\endhead
\bottomrule\noalign{}
\endlastfoot
issp\_01 & Which statement comes closes to expressing what you believe
about God? & \begin{minipage}[t]{\linewidth}\raggedright
\begin{enumerate}
\def\labelenumi{\arabic{enumi}.}
\tightlist
\item
  I don't believe in God
\item
  I don't know whether there is a God, and I don't believe there is any
  way to find out
\item
  I don't believe in a personal God, but I do believe in a Higher Power
  of some kind
\item
  I find myself believing in God some of the time, but not at others
\item
  While I have doubts, I feel that I do believe in God.
\item
  I know God really exists and I have no doubt about it.
\item
  Don't know
\end{enumerate}
\end{minipage} \\
issp\_02 & Here are some examples of how different people who do not
believe in God or gods identify themselves. If you had to pick a label,
which of these comes closest to how you identify yourself? &
\begin{minipage}[t]{\linewidth}\raggedright
\begin{enumerate}
\def\labelenumi{\arabic{enumi}.}
\tightlist
\item
  Spiritual but not religious
\item
  Seeker
\item
  Non-religious
\item
  Atheist
\item
  Agnostic
\item
  Humanist
\item
  Sceptic
\item
  Free thinker
\item
  Rationalist
\item
  Secular
\item
  Christian
\item
  Muslim
\item
  Jewish
\item
  Buddhist
\item
  Hindu
\item
  Daoist
\item
  Shinto
\item
  Confucian
\item
  Other (please specify)
\end{enumerate}
\end{minipage} \\
issp\_03 & Which of the following best describes your belief in god? &
\begin{minipage}[t]{\linewidth}\raggedright
\begin{enumerate}
\def\labelenumi{\arabic{enumi}.}
\tightlist
\item
  Generally speaking, I have always believed in God.
\item
  Generally speaking, I have never believed in God.
\item
  I believed in God in the past, but now I do not.
\item
  I did not believe in God in the past, but now I do.
\end{enumerate}
\end{minipage} \\
\end{longtable}

\section{Scoring}\label{scoring-18}

See the Belief Measures landing page for all belief measures variables
and their definitions.

\chapter{Religious Identity}\label{religious-identity}

\section{Measure}\label{measure-19}

\section{Implementation}\label{implementation-19}

\texttt{rid\_02\ only\ appears\ if\ rid\_01\ =\ Yes}

\begin{longtable}[]{@{}
  >{\raggedright\arraybackslash}p{(\columnwidth - 4\tabcolsep) * \real{0.0870}}
  >{\raggedright\arraybackslash}p{(\columnwidth - 4\tabcolsep) * \real{0.5652}}
  >{\raggedright\arraybackslash}p{(\columnwidth - 4\tabcolsep) * \real{0.3416}}@{}}
\toprule\noalign{}
\begin{minipage}[b]{\linewidth}\raggedright
Qlabel
\end{minipage} & \begin{minipage}[b]{\linewidth}\raggedright
Question
\end{minipage} & \begin{minipage}[b]{\linewidth}\raggedright
Response options
\end{minipage} \\
\midrule\noalign{}
\endhead
\bottomrule\noalign{}
\endlastfoot
rid\_01 & Do you regard yourself as belonging to a particular religion?
& Yes/No \\
rid\_02 & If yes, which? & \begin{minipage}[t]{\linewidth}\raggedright
\begin{enumerate}
\def\labelenumi{\arabic{enumi}.}
\item
  Protestant
\item
  Catholic
\item
  Orthodox (e.g.~Greek Orthodox, Russian Orthodox)
\item
  Sunni
\item
  Shiite
\item
  Buddhist
\item
  Confucian
\item
  Daoist
\item
  Hindu
\item
  Jewish
\item
  Shinto
\item
  Other (please specify)
\end{enumerate}
\end{minipage} \\
rid\_03 & Whilst growing up did your primary caregiver belong to a
particular religion? & Yes/No \\
rid\_04 & If yes, which? & \begin{minipage}[t]{\linewidth}\raggedright
\begin{enumerate}
\def\labelenumi{\arabic{enumi}.}
\item
  Protestant
\item
  Catholic
\item
  Orthodox (e.g.~Greek Orthodox, Russian Orthodox)
\item
  Sunni
\item
  Shiite
\item
  Buddhist
\item
  Confucian
\item
  Daoist
\item
  Hindu
\item
  Jewish
\item
  Shinto
\item
  Other (please specify)
\end{enumerate}
\end{minipage} \\
rid\_05 & Whilst growing up did an additional important caregiver belong
to a particular religion? & Yes/No \\
rid\_06 & If yes, which? & \begin{minipage}[t]{\linewidth}\raggedright
\begin{enumerate}
\def\labelenumi{\arabic{enumi}.}
\item
  Protestant
\item
  Catholic
\item
  Orthodox (e.g.~Greek Orthodox, Russian Orthodox)
\item
  Sunni
\item
  Shiite
\item
  Buddhist
\item
  Confucian
\item
  Daoist
\item
  Hindu
\item
  Jewish
\item
  Shinto
\item
  Other (please specify)
\end{enumerate}
\end{minipage} \\
\end{longtable}

\subsection{Scoring/ Coding}\label{scoring-coding}

See the Belief Measures landing page for all belief measures variables
and their definitions.

\section{}\label{section-3}

\chapter{Religious Practice}\label{religious-practice}

\textbf{Cluster:} Measuring Belief

\section{Measure}\label{measure-20}

\subsection*{Modifications}\label{modifications-17}
\addcontentsline{toc}{subsection}{Modifications}

\section{Implementation}\label{implementation-20}

\begin{longtable}[]{@{}
  >{\raggedright\arraybackslash}p{(\columnwidth - 4\tabcolsep) * \real{0.0695}}
  >{\raggedright\arraybackslash}p{(\columnwidth - 4\tabcolsep) * \real{0.6845}}
  >{\raggedright\arraybackslash}p{(\columnwidth - 4\tabcolsep) * \real{0.2406}}@{}}
\toprule\noalign{}
\begin{minipage}[b]{\linewidth}\raggedright
Qlabel
\end{minipage} & \begin{minipage}[b]{\linewidth}\raggedright
Question
\end{minipage} & \begin{minipage}[b]{\linewidth}\raggedright
Response options
\end{minipage} \\
\midrule\noalign{}
\endhead
\bottomrule\noalign{}
\endlastfoot
rp\_01 & Apart from weddings and funerals, about how often do you attend
religious services these days? &
\begin{minipage}[t]{\linewidth}\raggedright
\begin{enumerate}
\def\labelenumi{\arabic{enumi}.}
\tightlist
\item
  More than once a week
\item
  Once a week
\item
  Once a month
\item
  Only on special holy days
\item
  Once a year
\item
  Less often
\item
  Never, practically never
\end{enumerate}
\end{minipage} \\
rp\_02 & Apart from weddings and funerals, about how often do you pray?
& \begin{minipage}[t]{\linewidth}\raggedright
\begin{enumerate}
\def\labelenumi{\arabic{enumi}.}
\tightlist
\item
  Several times a day
\item
  Once a day
\item
  Several times each week
\item
  Only when attending religious services
\item
  Only on special holy days
\item
  Once a year
\item
  Less often
\item
  Never, practically never
\end{enumerate}
\end{minipage} \\
rp\_03 & Do you have in your home a shrine, altar, or a religious object
on display such as a (BIBLE OR CROSS/ COUNTRY SPECIFIC ITEM)? &
\begin{minipage}[t]{\linewidth}\raggedright
\begin{enumerate}
\def\labelenumi{\arabic{enumi}.}
\tightlist
\item
  Yes, for religious reasons
\item
  Yes, for non-religious reasons
\item
  No
\end{enumerate}
\end{minipage} \\
\end{longtable}

\textbf{Country Specific Examples}

\emph{Brazil}

rp\_03: crucifix or an image of Iemanja

\emph{China}

rp\_03: an ancestor tablet, ~censer, Buddha statue, or statue of a
deity?

\emph{Denmark}

rp\_03: a Bible or cross?

\emph{Japan}

rp\_03: butsudan, ihai, or kamidana?

\emph{UK}

rp\_03: cross, icon, mezuzah, or Bible?

\emph{USA}

rp\_03: a cross, icon, mezuzah, Bible, or retablo?

\subsection{Scoring/ Coding}\label{scoring-coding-1}

See the General Belief landing page for all belief measures variables
and their definitions.

\chapter{Supernatural Belief}\label{supernatural-belief-1}

\textbf{Cluster:} Belief Measures

\section{Measure}\label{measure-21}

\subsection*{Modifications}\label{modifications-18}
\addcontentsline{toc}{subsection}{Modifications}

\section{Implementation}\label{implementation-21}

\subsection*{Question wording}\label{question-wording-18}
\addcontentsline{toc}{subsection}{Question wording}

Participants read the following text:

\emph{Please read each of the following statements carefully and state
to what extent you agree. There are no right or wrong answers and your
responses remain anonymous.}

\subsection{Items}\label{items-18}

\begin{longtable}[]{@{}
  >{\raggedright\arraybackslash}p{(\columnwidth - 2\tabcolsep) * \real{0.2361}}
  >{\raggedright\arraybackslash}p{(\columnwidth - 2\tabcolsep) * \real{0.7639}}@{}}
\toprule\noalign{}
\begin{minipage}[b]{\linewidth}\raggedright
Qlabel
\end{minipage} & \begin{minipage}[b]{\linewidth}\raggedright
question
\end{minipage} \\
\midrule\noalign{}
\endhead
\bottomrule\noalign{}
\endlastfoot
snb\_01 & There is some sort of life after death \\
snb\_02 & Sometime after I die, I expect that I'll be born again in
another body. \\
snb\_03 & The positions of the stars and planets affect people's
lives \\
snb\_04 & Some people have mystical powers (e.g.~to heal, harm, or bring
good luck) \\
snb\_05 & Some objects have mystical powers (e.g.~to heal, harm, or
bring good luck) \\
snb\_06 & There are underlying forces of good and evil in this world. \\
snb\_07 & There exists a universal spirit or life force. \\
snb\_08 & There is a power in the universe that causes good things to
happen to people who behave morally and bad things to happen to people
who behave immorally. \\
snb\_09 & Most significant life events are meant to be and happen for a
reason. \\
snb\_10 & Supernatural beings of some kind exist \\
snb\_11 & There exist supernatural beings that are good/kind
(e.g.~COUNTRY SPECIFIC) \\
snb\_12 & There exist supernatural beings that are harmful. (e.g COUNTRY
SPECIFIC) \\
snb\_13 & Some people will be punished after they die \\
snb\_14 & Some people will be rewarded after they die \\
snb\_15 & Making other people envious of you can cause illness or
misfortune. \\
snb\_16 & Have you ever felt as though you were connected to a powerful
spiritual force? \\
snb\_17 & Do you carry any objects for luck or protection? \\
\end{longtable}

\textbf{Country Specific Examples}

\emph{Brazil}

snb\_11: e.g.~angels, nature spirits

snb\_12: e.g.~demons, ghosts

\emph{China}

snb\_11: e.g.~ancestor spirits

snb\_12: e.g.~ghosts

\emph{Denmark}

snb\_11: e.g.~angels, nature spirits

snb\_12: e.g.~demons, ghosts

\emph{Japan}

snb\_11: e.g.~angels, spirits

snb\_12: e.g.~ ghosts, monsters

\emph{UK}

snb\_11: e.g.~angels, nature spirits

snb\_12: e.g.~demons, ghosts

\emph{USA}

snb\_11: e.g.~angels, nature spirits

snb\_12: e.g.~demons, ghosts

\section{Coding}\label{coding-18}

\textbf{Items snb\_01 - snb\_15 use our standard response scale for
agreement}

\begin{longtable}[]{@{}
  >{\raggedright\arraybackslash}p{(\columnwidth - 12\tabcolsep) * \real{0.1429}}
  >{\raggedright\arraybackslash}p{(\columnwidth - 12\tabcolsep) * \real{0.1429}}
  >{\raggedright\arraybackslash}p{(\columnwidth - 12\tabcolsep) * \real{0.1429}}
  >{\raggedright\arraybackslash}p{(\columnwidth - 12\tabcolsep) * \real{0.1429}}
  >{\raggedright\arraybackslash}p{(\columnwidth - 12\tabcolsep) * \real{0.1429}}
  >{\raggedright\arraybackslash}p{(\columnwidth - 12\tabcolsep) * \real{0.1429}}
  >{\raggedright\arraybackslash}p{(\columnwidth - 12\tabcolsep) * \real{0.1429}}@{}}
\toprule\noalign{}
\endhead
\bottomrule\noalign{}
\endlastfoot
\textbf{1} & \textbf{2} & \textbf{3} & \textbf{4} & \textbf{5} &
\textbf{6} & \textbf{7} \\
strongly disagree & moderately disagree & slightly disagree & neither
agree nor disagree & slightly agree & moderately agree & strongly
agree \\
\end{longtable}

\textbf{Item snb\_16 uses the following bespoke scale}

\begin{longtable}[]{@{}
  >{\raggedright\arraybackslash}p{(\columnwidth - 4\tabcolsep) * \real{0.2500}}
  >{\raggedright\arraybackslash}p{(\columnwidth - 4\tabcolsep) * \real{0.4861}}
  >{\raggedright\arraybackslash}p{(\columnwidth - 4\tabcolsep) * \real{0.2639}}@{}}
\toprule\noalign{}
\begin{minipage}[b]{\linewidth}\raggedright
1
\end{minipage} & \begin{minipage}[b]{\linewidth}\raggedright
2
\end{minipage} & \begin{minipage}[b]{\linewidth}\raggedright
3
\end{minipage} \\
\midrule\noalign{}
\endhead
\bottomrule\noalign{}
\endlastfoot
Yes, I've had an experience like this. & I've had an experience like
this, but I didn't associate it with a spiritual force. & No, I've never
had an experience like this. \\
\end{longtable}

\textbf{Item snb\_17 is a Yes/No response}

\subsection{Scoring}\label{scoring-19}

See the General Belief landing page for all belief measures variables
and their definitions.

\chapter{General Belief}\label{general-belief}

\textbf{Cluster:} Belief Measures

\section{Measure}\label{measure-22}

\subsection*{Modifications}\label{modifications-19}
\addcontentsline{toc}{subsection}{Modifications}

\section{Implementation}\label{implementation-22}

\subsection*{Question wording}\label{question-wording-19}
\addcontentsline{toc}{subsection}{Question wording}

Participants read the following text:

\emph{``The following section will list a number of statements regarding
your attitudes towards God, religion, and belief. Please read each and
state to what extent you agree. There are no right or wrong answers and
your responses remain anonymous.''}

\subsection{Items}\label{items-19}

\begin{longtable}[]{@{}
  >{\raggedright\arraybackslash}p{(\columnwidth - 2\tabcolsep) * \real{0.1972}}
  >{\raggedright\arraybackslash}p{(\columnwidth - 2\tabcolsep) * \real{0.8028}}@{}}
\toprule\noalign{}
\begin{minipage}[b]{\linewidth}\raggedright
Qlabel
\end{minipage} & \begin{minipage}[b]{\linewidth}\raggedright
question
\end{minipage} \\
\midrule\noalign{}
\endhead
\bottomrule\noalign{}
\endlastfoot
belief\_01 & I believe that God exists \\
belief\_02 & It is not possible to know if God exists \\
conf\_01 & I am confident that my beliefs about God\textquotesingle s
existence are the right ones. \\
ar\_01 & We would all be better off if people left religion behind. \\
apth\_01 & Whether or not God exists is a question that doesn't interest
me much \\
apth\_02 & Whether or not there is an ultimate purpose to life is a
question that doesn't interest me much. \\
agn\_01 & I don't know whether there is a God, and I don't believe there
is any way to find out. \\
mean\_01 & For most things in the world, we will never be able to
discover the real objective truth. \\
\end{longtable}

\section{Coding}\label{coding-19}

This scale uses our standard response scale for agreement

\begin{longtable}[]{@{}
  >{\raggedright\arraybackslash}p{(\columnwidth - 12\tabcolsep) * \real{0.1429}}
  >{\raggedright\arraybackslash}p{(\columnwidth - 12\tabcolsep) * \real{0.1429}}
  >{\raggedright\arraybackslash}p{(\columnwidth - 12\tabcolsep) * \real{0.1429}}
  >{\raggedright\arraybackslash}p{(\columnwidth - 12\tabcolsep) * \real{0.1429}}
  >{\raggedright\arraybackslash}p{(\columnwidth - 12\tabcolsep) * \real{0.1429}}
  >{\raggedright\arraybackslash}p{(\columnwidth - 12\tabcolsep) * \real{0.1429}}
  >{\raggedright\arraybackslash}p{(\columnwidth - 12\tabcolsep) * \real{0.1429}}@{}}
\toprule\noalign{}
\endhead
\bottomrule\noalign{}
\endlastfoot
\textbf{1} & \textbf{2} & \textbf{3} & \textbf{4} & \textbf{5} &
\textbf{6} & \textbf{7} \\
strongly disagree & moderately disagree & slightly disagree & neither
agree nor disagree & slightly agree & moderately agree & strongly
agree \\
\end{longtable}

\section{Scoring}\label{scoring-20}

See the Belief Measures landing page for all belief measures variables
and their definitions.

\chapter{Atheist Membership}\label{atheist-membership}

\textbf{Cluster:} Measuring Belief

\section{Measure}\label{measure-23}

\subsection*{Modifications}\label{modifications-20}
\addcontentsline{toc}{subsection}{Modifications}

\section{Implementation}\label{implementation-23}

\begin{longtable}[]{@{}
  >{\raggedright\arraybackslash}p{(\columnwidth - 4\tabcolsep) * \real{0.1944}}
  >{\raggedright\arraybackslash}p{(\columnwidth - 4\tabcolsep) * \real{0.6111}}
  >{\raggedright\arraybackslash}p{(\columnwidth - 4\tabcolsep) * \real{0.1944}}@{}}
\toprule\noalign{}
\begin{minipage}[b]{\linewidth}\raggedright
Qlabel
\end{minipage} & \begin{minipage}[b]{\linewidth}\raggedright
Question
\end{minipage} & \begin{minipage}[b]{\linewidth}\raggedright
Response options
\end{minipage} \\
\midrule\noalign{}
\endhead
\bottomrule\noalign{}
\endlastfoot
am\_01 & Are you currently a member of any atheist, secularist,
humanist, or similar organization at a national or local level? & Y/N \\
\end{longtable}

\section{Scoring}\label{scoring-21}

See the Belief Measures landing page for all belief measures variables
and their definitions.

\part{References}

\phantomsection\label{refs}
\begin{CSLReferences}{1}{0}
\bibitem[\citeproctext]{ref-baimel2022}
Baimel, A., Apicella, C., Atkinson, Q., Bolyanatz, A., Cohen, E.,
Handley, C., \ldots{} Purzycki, B. (2022). Material insecurity predicts
greater commitment to moralistic and less commitment to local deities: A
cross-cultural investigation. \emph{Religion, Brain \& Behavior},
\emph{12}(1-2), 4--17.
\url{https://doi.org/10.1080/2153599X.2021.2006287}

\bibitem[\citeproctext]{ref-baron-cohen2004}
Baron-Cohen, S., \& Wheelwright, S. (2004). The {Empathy Quotient}: {An
Investigation} of {Adults} with {Asperger Syndrome} or {High Functioning
Autism}, and {Normal Sex Differences}. \emph{Journal of Autism and
Developmental Disorders}, \emph{34}(2), 163--175.
\url{https://doi.org/10.1023/B:JADD.0000022607.19833.00}

\bibitem[\citeproctext]{ref-frederick2005}
Frederick, S. (2005). Cognitive {Reflection} and {Decision Making}.
\emph{Journal of Economic Perspectives}, \emph{19}(4), 25--42.
\url{https://doi.org/10.1257/089533005775196732}

\bibitem[\citeproctext]{ref-gelfand1999}
Gelfand, M. J., \& Realo, A. (1999). Individualism-collectivism and
accountability in intergroup negotiations. \emph{Journal of Applied
Psychology}, \emph{84}(5), 721--736.
\url{https://doi.org/10.1037/0021-9010.84.5.721}

\bibitem[\citeproctext]{ref-hart2015}
Hart, C. M., Ritchie, T. D., Hepper, E. G., \& Gebauer, J. E. (2015).
The {Balanced Inventory} of {Desirable Responding Short Form}
({BIDR-16}). \emph{SAGE Open}, \emph{5}(4), 2158244015621113.
\url{https://doi.org/10.1177/2158244015621113}

\bibitem[\citeproctext]{ref-lanman2017}
Lanman, J. A., \& Buhrmester, M. D. (2017). Religious actions speak
louder than words: Exposure to credibility-enhancing displays predicts
theism. \emph{Religion, Brain \& Behavior}, \emph{7}(1), 3--16.
\url{https://doi.org/10.1080/2153599X.2015.1117011}

\bibitem[\citeproctext]{ref-marks1973}
Marks, D. F. (1973). Visual {Imagery Differences} in the {Recall} of
{Pictures}. \emph{British Journal of Psychology}, \emph{64}(1), 17--24.
\url{https://doi.org/10.1111/j.2044-8295.1973.tb01322.x}

\bibitem[\citeproctext]{ref-mcdermott2001}
McDermott, M. R. (2001). Rebelliousness. In \emph{Motivational styles in
everyday life: {A} guide to reversal theory} (pp. 167--185).
{Washington, DC, US}: {American Psychological Association}.
\url{https://doi.org/10.1037/10427-009}

\bibitem[\citeproctext]{ref-mclain2009}
McLain, D. L. (2009). Evidence of the properties of an ambiguity
tolerance measure: The {Multiple Stimulus Types Ambiguity Tolerance
Scale-II} ({MSTAT-II}). \emph{Psychological Reports}, \emph{105}(3 Pt
1), 975--988. \url{https://doi.org/10.2466/PR0.105.3.975-988}

\bibitem[\citeproctext]{ref-neave2015}
Neave, N., Jackson, R., Saxton, T., \& Hönekopp, J. (2015). The
influence of anthropomorphic tendencies on human hoarding behaviours.
\emph{Personality and Individual Differences}, \emph{72}, 214--219.
\url{https://doi.org/10.1016/j.paid.2014.08.041}

\bibitem[\citeproctext]{ref-stahl2016}
Ståhl, T., Zaal, M. P., \& Skitka, L. J. (2016). Moralized rationality:
{Relying} on logic and evidence in the formation and evaluation of
belief can be seen as a moral issue. \emph{PLoS ONE}, \emph{11}.
\url{https://doi.org/10.1371/journal.pone.0166332}

\bibitem[\citeproctext]{ref-stanovich1989}
Stanovich, K. E. (1989). Implicit {Philosophies} of {Mind}: {The Dualism
Scale} and {Its Relation} to {Religiosity} and {Belief} in {Extrasensory
Perception}. \emph{The Journal of Psychology}, \emph{123}(1), 5--23.
\url{https://doi.org/10.1080/00223980.1989.10542958}

\bibitem[\citeproctext]{ref-steger2006}
Steger, M. F., Frazier, P., Oishi, S., \& Kaler, M. (2006). The
{Meaning} in {Life Questionnaire}: {Assessing} the presence of and
search for meaning in life. \emph{Journal of Counseling Psychology},
\emph{53}(1), 80--93. \url{https://doi.org/10.1037/0022-0167.53.1.80}

\bibitem[\citeproctext]{ref-thomson2016}
Thomson, K. S., \& Oppenheimer, D. M. (2016). Investigating an alternate
form of the cognitive reflection test. \emph{Judgment and Decision
Making}, \emph{11}, 99--113.

\bibitem[\citeproctext]{ref-wakabayashi2006}
Wakabayashi, A., Baron-Cohen, S., Wheelwright, S., Goldenfeld, N.,
Delaney, J., Fine, D., \ldots{} Weil, L. (2006). Development of short
forms of the {Empathy Quotient} ({EQ-Short}) and the {Systemizing
Quotient} ({SQ-Short}). \emph{Personality and Individual Differences},
\emph{41}(5), 929--940. \url{https://doi.org/10.1016/j.paid.2006.03.017}

\bibitem[\citeproctext]{ref-willard2017}
Willard, A. K., \& Cingl, L. (2017). Testing theories of secularization
and religious belief in the {Czech Republic} and {Slovakia}.
\emph{Evolution and Human Behavior}, \emph{38}(5), 604--615.
\url{https://doi.org/10.1016/j.evolhumbehav.2017.01.002}

\bibitem[\citeproctext]{ref-willard2020}
Willard, A. K., Cingl, L., \& Norenzayan, A. (2020). Cognitive {Biases}
and {Religious Belief}: {A Path Model Replication} in the {Czech
Republic} and {Slovakia With} a {Focus} on {Anthropomorphism}.
\emph{Social Psychological and Personality Science}, \emph{11}(1),
97--106. \url{https://doi.org/10.1177/1948550619841629}

\end{CSLReferences}

\chapter*{References}\label{references-1}
\addcontentsline{toc}{chapter}{References}

\markboth{References}{References}

\phantomsection\label{refs}
\begin{CSLReferences}{1}{0}
\bibitem[\citeproctext]{ref-baimel2022}
Baimel, A., Apicella, C., Atkinson, Q., Bolyanatz, A., Cohen, E.,
Handley, C., \ldots{} Purzycki, B. (2022). Material insecurity predicts
greater commitment to moralistic and less commitment to local deities: A
cross-cultural investigation. \emph{Religion, Brain \& Behavior},
\emph{12}(1-2), 4--17.
\url{https://doi.org/10.1080/2153599X.2021.2006287}

\bibitem[\citeproctext]{ref-baron-cohen2004}
Baron-Cohen, S., \& Wheelwright, S. (2004). The {Empathy Quotient}: {An
Investigation} of {Adults} with {Asperger Syndrome} or {High Functioning
Autism}, and {Normal Sex Differences}. \emph{Journal of Autism and
Developmental Disorders}, \emph{34}(2), 163--175.
\url{https://doi.org/10.1023/B:JADD.0000022607.19833.00}

\bibitem[\citeproctext]{ref-frederick2005}
Frederick, S. (2005). Cognitive {Reflection} and {Decision Making}.
\emph{Journal of Economic Perspectives}, \emph{19}(4), 25--42.
\url{https://doi.org/10.1257/089533005775196732}

\bibitem[\citeproctext]{ref-gelfand1999}
Gelfand, M. J., \& Realo, A. (1999). Individualism-collectivism and
accountability in intergroup negotiations. \emph{Journal of Applied
Psychology}, \emph{84}(5), 721--736.
\url{https://doi.org/10.1037/0021-9010.84.5.721}

\bibitem[\citeproctext]{ref-hart2015}
Hart, C. M., Ritchie, T. D., Hepper, E. G., \& Gebauer, J. E. (2015).
The {Balanced Inventory} of {Desirable Responding Short Form}
({BIDR-16}). \emph{SAGE Open}, \emph{5}(4), 2158244015621113.
\url{https://doi.org/10.1177/2158244015621113}

\bibitem[\citeproctext]{ref-lanman2017}
Lanman, J. A., \& Buhrmester, M. D. (2017). Religious actions speak
louder than words: Exposure to credibility-enhancing displays predicts
theism. \emph{Religion, Brain \& Behavior}, \emph{7}(1), 3--16.
\url{https://doi.org/10.1080/2153599X.2015.1117011}

\bibitem[\citeproctext]{ref-marks1973}
Marks, D. F. (1973). Visual {Imagery Differences} in the {Recall} of
{Pictures}. \emph{British Journal of Psychology}, \emph{64}(1), 17--24.
\url{https://doi.org/10.1111/j.2044-8295.1973.tb01322.x}

\bibitem[\citeproctext]{ref-mcdermott2001}
McDermott, M. R. (2001). Rebelliousness. In \emph{Motivational styles in
everyday life: {A} guide to reversal theory} (pp. 167--185).
{Washington, DC, US}: {American Psychological Association}.
\url{https://doi.org/10.1037/10427-009}

\bibitem[\citeproctext]{ref-mclain2009}
McLain, D. L. (2009). Evidence of the properties of an ambiguity
tolerance measure: The {Multiple Stimulus Types Ambiguity Tolerance
Scale-II} ({MSTAT-II}). \emph{Psychological Reports}, \emph{105}(3 Pt
1), 975--988. \url{https://doi.org/10.2466/PR0.105.3.975-988}

\bibitem[\citeproctext]{ref-neave2015}
Neave, N., Jackson, R., Saxton, T., \& Hönekopp, J. (2015). The
influence of anthropomorphic tendencies on human hoarding behaviours.
\emph{Personality and Individual Differences}, \emph{72}, 214--219.
\url{https://doi.org/10.1016/j.paid.2014.08.041}

\bibitem[\citeproctext]{ref-stahl2016}
Ståhl, T., Zaal, M. P., \& Skitka, L. J. (2016). Moralized rationality:
{Relying} on logic and evidence in the formation and evaluation of
belief can be seen as a moral issue. \emph{PLoS ONE}, \emph{11}.
\url{https://doi.org/10.1371/journal.pone.0166332}

\bibitem[\citeproctext]{ref-stanovich1989}
Stanovich, K. E. (1989). Implicit {Philosophies} of {Mind}: {The Dualism
Scale} and {Its Relation} to {Religiosity} and {Belief} in {Extrasensory
Perception}. \emph{The Journal of Psychology}, \emph{123}(1), 5--23.
\url{https://doi.org/10.1080/00223980.1989.10542958}

\bibitem[\citeproctext]{ref-steger2006}
Steger, M. F., Frazier, P., Oishi, S., \& Kaler, M. (2006). The
{Meaning} in {Life Questionnaire}: {Assessing} the presence of and
search for meaning in life. \emph{Journal of Counseling Psychology},
\emph{53}(1), 80--93. \url{https://doi.org/10.1037/0022-0167.53.1.80}

\bibitem[\citeproctext]{ref-thomson2016}
Thomson, K. S., \& Oppenheimer, D. M. (2016). Investigating an alternate
form of the cognitive reflection test. \emph{Judgment and Decision
Making}, \emph{11}, 99--113.

\bibitem[\citeproctext]{ref-wakabayashi2006}
Wakabayashi, A., Baron-Cohen, S., Wheelwright, S., Goldenfeld, N.,
Delaney, J., Fine, D., \ldots{} Weil, L. (2006). Development of short
forms of the {Empathy Quotient} ({EQ-Short}) and the {Systemizing
Quotient} ({SQ-Short}). \emph{Personality and Individual Differences},
\emph{41}(5), 929--940. \url{https://doi.org/10.1016/j.paid.2006.03.017}

\bibitem[\citeproctext]{ref-willard2017}
Willard, A. K., \& Cingl, L. (2017). Testing theories of secularization
and religious belief in the {Czech Republic} and {Slovakia}.
\emph{Evolution and Human Behavior}, \emph{38}(5), 604--615.
\url{https://doi.org/10.1016/j.evolhumbehav.2017.01.002}

\bibitem[\citeproctext]{ref-willard2020}
Willard, A. K., Cingl, L., \& Norenzayan, A. (2020). Cognitive {Biases}
and {Religious Belief}: {A Path Model Replication} in the {Czech
Republic} and {Slovakia With} a {Focus} on {Anthropomorphism}.
\emph{Social Psychological and Personality Science}, \emph{11}(1),
97--106. \url{https://doi.org/10.1177/1948550619841629}

\end{CSLReferences}


\backmatter

\end{document}
